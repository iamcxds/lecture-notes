\documentclass[12pt]{book}
\usepackage{amssymb}
\usepackage{amsmath}
\usepackage{amsfonts}
\usepackage{tikz-cd}
\usepackage{stmaryrd}
\usepackage{hyperref}
\setlength{\evensidemargin}{0.25in}
\setlength{\oddsidemargin}{0.25in}
\setlength{\textwidth}{6in}
\parskip0.2em

\newtheorem{theorem}{Theorem}[section]
\newtheorem{lemma}[theorem]{Lemma}
\newtheorem{proposition}[theorem]{Proposition}
\newtheorem{corollary}[theorem]{Corollary}
\newtheorem{definition}[theorem]{Definition}
\newtheorem{conj}[theorem]{Conjecture}
\newtheorem{assumption}[theorem]{Assumption}
\newtheorem{property}[theorem]{Property}
\newtheorem{remark}[theorem]{Remark}
\newtheorem{example}[theorem]{Example}
\newtheorem{exercise}[theorem]{Exercise}
\numberwithin{equation}{section}
\allowdisplaybreaks[1]

%Peng's command
\newcommand{\MW}{Milnor-Witt\ }
\newcommand{\rMW}{\mathrm{MW}}
\newcommand{\KMW}{\mathrm{K}^\mathrm{MW}}
\newcommand{\KM}{\mathrm{K}^\mathrm{M}}
\newcommand{\sKMW}{\mathscr{K}^\mathrm{MW}}
\newcommand{\tbb}[1]{\widetilde{\mathbb{#1}}}
\newcommand{\wt}[1]{\widetilde{#1}}
\newcommand{\Spec}{\mathrm{Spec}\ }
\newcommand{\af}{\mathbb{A}}
\newcommand{\afnz}[1]{\mathbb{A}^{#1}\setminus \{0\}}
\newcommand{\nonZero}{\setminus \{0\}}
\newcommand{\cO}{\mathcal{O}}
\newcommand{\bZ}{\mathbb{Z}}
\newcommand{\bfZ}{\mathbf{Z}}
\newcommand{\tbZ}{\tbb{Z}}
\newcommand{\inv}{^{-1}}%
\newcommand{\Gm}{\mathbb{G}_m}
\newcommand{\tGm}{\tbb{G}_m}
\newcommand{\Sp}{\mathrm{Sp}}
\newcommand{\GL}{\mathrm{GL}}
\newcommand{\SL}{\mathrm{SL}}
\newcommand{\MSp}{\mathrm{MSp}}
\newcommand{\BSp}{\mathrm{BSp}}
\newcommand{\SH}{\mathcal{SH}}
\newcommand{\Da}{D_{\af^1}}
\newcommand{\Daba}[1]{D(Ab_{\af^1}(#1))}
\newcommand{\DM}{\mathrm{DM}}
\newcommand{\DMt}{\widetilde{\mathrm{DM}}}
\newcommand{\M}{\mathrm{M}}
\newcommand{\Mt}{\wt{\mathrm{M}}}
\newcommand{\Meta}{\wt{\mathrm{M}}_{\eta}}
\newcommand{\HH}{\mathrm{H}_{\rMW}}
\newcommand{\Heta}{\mathrm{H}_{\eta}}
\newcommand{\sBra}[1]{\left[#1\right]}
\newcommand{\aBra}[1]{\left<#1\right>}
\newcommand{\llBra}[1]{\llbracket #1 \rrbracket
}
\newcommand{\rHom}{\mathrm{Hom}}
\newcommand{\Ker}{\mathrm{Ker}}
\newcommand{\rFun}{\mathrm{Fun}}
\newcommand{\Spc}{\mathcal{S}pc}
\newcommand{\gy}{\rm{Gysin}}
\newcommand{\xr}[1]{\xrightarrow{#1}}

\newcommand{\AZ}{\mathbb{A}\mathcal{Z}}
\newcommand{\bfZa}{\mathbf{Z}_{\mathbb{A}^1}}
\newcommand{\tauO}{\tau^{\odot}}

\raggedbottom

\title{Seeing the Mountain}
\author{Keyao Peng}



\begin{document}
\begin{titlepage}
\begin{center}
{\huge\bfseries Seeing the Mountain \\}
% {\Large A Journey Through Post-Modern Geometry \\}
 % ----------------------------------------------------------------
 \vspace{1.5cm}
 {\Large\bfseries Keyao Peng}\\[5pt]
 keyao.peng@ube.fr\\[14pt]
  % ----------------------------------------------------------------
 \vspace{2cm}
 % ----------------------------------------------------------------
\includegraphics[width=\textwidth]{cover.png}\\[5pt]
 \vfill
{Sep 2025}
\end{center}
\end{titlepage}
\clearpage

\thispagestyle{empty}
\newlength{\longest}
\settowidth\longest{\huge\itshape Seeing the mountain still as a mountain.}
\null\vfill

\centering
\parbox{\longest}{%
  \raggedright{\huge\itshape%
  Seeing the mountain as the mountain; \\ 
  Seeing the mountain as not a mountain; \\ 
  Seeing the mountain as still a mountain. \par\bigskip
  }   
  \raggedleft\Large\MakeUppercase{Qingyuan Xingsi}\par%
}

\vfill\vfill

\clearpage

\tableofcontents

\chapter{Introduction}
\section{What's Geometry?}

\section{What's a Space?}

\section{What's Physics about?}



\bibliographystyle{plain}
\bibliography{references}
\end{document}
