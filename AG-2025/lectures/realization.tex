\chapter{Realization and Invariant}\label{chap:realization} % (fold)

We have seen how to define a generalized space via sheaf, which is data of testing spaces map into it. In this chapter we reconstruct space from the sheaf. That is a (fully) faithful\footnote{Fully means surjective on morphism, faithful means injective on morphism.} functor of \emph{realization} $R: \Sh(\cC) \to \mathcal{S} $ to some concrete category of some kind of spaces.

On the other hand, sometimes it is also useful to construct functor $I: \Sh(\cC) \to \mathcal{V} $ to a simpler category $\mathcal{V}$ (value), even thought we lose some information. This is called \emph{invariant}.

\section{Relative Topos/Geometric morphism of Topos}
We have seen a topos $\mathcal{T}$ (category of sheaf) as a collection of generalized spaces. Sometimes we can seen it as generalized spaces relative certain space $X$, in that case $\mathcal{T}_X$ also reflex the geometry of $X$.

\begin{example}
  For topological spaces $X$, we have seen $\SH(X) \cong \mathrm{Et}/ X$ étale space over $X$. Then for any continuous map $f:X\to Y$, we have the following adjoint functors:
  \[
    f^{-1}: \Sh(Y) \leftrightarrows \Sh(X): f_*
  \]
  Where for $ U \in \Op(Y)$, we ask $f^{-1}(\yo (U)) := \yo (f^{-1}(U)) $. Then by the property of adjoint functors we have for $\cF\in \Sh(X)$ 
  \[
    f_*\cF(U) = \Hom_{\Sh(Y)}(\yo(U),f_*\cF ) \cong \Hom_{\Sh(X)}(f^{-1}(\yo(U)),\cF ) = \cF(f^{-1}(U))
  \]
  To define $ f^{-1}\cG$ for general $\cG \in \Sh(Y)$, recall $ \cG \cong \int_{U\in \Op(Y)} \cG(U) \times \yo(U)$, then since $f^{-1}$ preserve colimit, we have
  \[
    f^{-1}\cG \cong L\int_{U\in \Op(X)} \cG(U) \times f^{-1}(\yo(U))
  \]

  \begin{exercise}*
    Show that the presheaf $\int_{U\in \Op(X)} \cG(U) \times f^{-1}(\yo(U)) \in \PSh(X)$ is given by $ V\in \Op(X) \mapsto \Colim_{f(V) \subset U \in \Op(Y)}\cG(U)$  
  \end{exercise}

  If we look at some example of continuous map to a point $p:X\to * $.

  Then we consider a point $x\in X$, viewed as a map $x: * \to X$.
\end{example}

\begin{example}[Slice Category]
  Let $S\in \Sh(\cC)$, then the slice category $ \Sh(\cC)/S $ ($\cong \Sh(\int_\cC S)$ exercise) consists of morphisms $p :X \to S$. For a morphism $f:S\to T$, we can simply define $f^*(p:X\to S):= f^*p: X\times_S T \to T $. In fact $f^*$ has both left and right adjoint functors $f_\#,f_*$. 
\end{example} 

point of topos

points and prime ideals.

classifying topos of local ring
\section{Ringed Space}

Recall for $\cD$ cocomplete, if we have a realization functor on test objects $R:\cC \to \cD$, we can extend it to all presheaf $R:\PSh(\cC)\to \cD$ by using category of elements 

\[
  R:\cF \in \PSh(\cC) \mapsto \int_\cC \cF\times R \in \cD
\]
This means we only need to consider the realization functor at level of test objects $\cC$.

Recall for the simplicial set, smooth set, algebraic set, we can have realization functor in topological space $\Top$:

\begin{example}
 \begin{enumerate}
  \item For simplex category, we can take the underlying space $|-|:\Delta \to \Top$, thus we have $|-|: \PSh(\Delta) \to \Top$
  \item Similarly for Cartesian spaces, we can take the underlying space $|-|:\Cart \to \Top$, thus we have $|-|: \PSh(\Cart) \to \Top$
  \item For the category of ring $\Ring$, we can take spectrum $\Spec:\Ring^\op \to \Top$: The spectrum of a ring $A$ is the topological space $\Spec A$ whose points are prime ideals of $A$, topological base is given by $f\in A, D(f):= \{\mathfrak{p}\in \Spec A \mid f\notin \mathfrak{p}\}$. Thus we have $\Spec: \PSh(\Ring^\op) \to \Top$.
 \end{enumerate} 
\end{example}



But this is not yet what we want: the realization functor is not fully, that is, for example there are much more continuous morphisms that smooth morphisms for manifold. To solve this problem, we need to update the target category.

The spaces we consider they all have function ring on it, and the morphisms of function rings will reflect the property of morphisms of space

\begin{exercise}
Consider a map $f:\bR \to \bR^2$,
\begin{enumerate}
\item if $f$ is continuous, this induces a ring homomorphism $f^\#:C(\bR^2)\to C(\bR)$.
\item if $f$ is smooth, this induces a ring homomorphism $f^\#:C^\infty(\bR^2)\to C^\infty(\bR)$.
\item if $f$ is algebraic, this induces a ring homomorphism $f^\#:\bR[x,y]\to \bR[t]$.
\end{enumerate}
\end{exercise}
This motivates us of following definition:
\begin{definition}[Locally Ringed Space]
A \emph{locally ringed space} is a pair $(X, \mathcal{O}_X) \in LRS$ where:
\begin{itemize}
\item $X$ is a topological space,
\item $\mathcal{O}_X \in \Sh(X,\Ring)$ is a sheaf of rings on $X$, called \emph{structure sheaf},
\item for every point $x \in X$, the stalk $\mathcal{O}_{X,x}$ is a local ring (i.e., it has a unique maximal ideal).
\end{itemize}

A morphism of locally ringed spaces
\[
(f, f^\#) : (X, \mathcal{O}_X) \longrightarrow (Y, \mathcal{O}_Y)
\]
consists of:
\begin{itemize}
\item a continuous map $f : X \to Y$,
\item a morphism of sheaves of rings
\[
    f^\# : \mathcal{O}_Y \longrightarrow f_* \mathcal{O}_X,
\]
such that for every $x \in X$, the induced map on stalks
\[
f^\#_x : \mathcal{O}_{Y,f(x)} \longrightarrow \mathcal{O}_{X,x}
\]
is a local homomorphism (i.e., it sends the maximal ideal of $\mathcal{O}_{Y,f(x)}$ into the maximal ideal of $\mathcal{O}_{X,x}$).
\end{itemize}

\end{definition}

Even thought we first ask $f$ to be continuous, to define $f^\#$ it turns out $f$ have to be the map preserve some structure.

\begin{example}
\begin{enumerate}
\item \textbf{Topological Space with Continuous Functions:}  
For any topological space $X$, $(X, C_X)$, where $C_X$ is the sheaf of continuous real-valued functions, is a locally ringed space. Each stalk is a local ring of germs of continuous functions.

\item \textbf{Smooth Manifold:}  
Let $M$ be a smooth manifold. The pair $(M, C^\infty_M)$, where $C^\infty_M$ is the sheaf of smooth real-valued functions on $M$, is a locally ringed space. Each stalk consists of germs of smooth functions at a point, forming a local ring.

For smooth manifolds $M$ and $N$ and a smooth map $f : M \to N$, we have:
\[
  (f,f^\#):(M, \mathcal{C}^\infty_M) \longrightarrow (N, \mathcal{C}^\infty_N)
\]
where $f^\#$ sends a smooth function $h$ on $N$ to $h \circ f$ on $M$.

\item \textbf{Complex Analytic Space:}  
For a complex manifold $X$, the pair $(X, \mathcal{O}_X)$, where $\mathcal{O}_X$ is the sheaf of holomorphic functions, is a locally ringed space. Each stalk is a local ring of germs of holomorphic functions.

For complex manifolds $X$ and $Y$ and a holomorphic map $f : X \to Y$, the morphism:
\[
(f,f^\#):(X, \mathcal{O}_X) \longrightarrow (Y, \mathcal{O}_Y)
\]
where $f^\#$ sends a holomorphic function $h$ on $Y$ to $h \circ f$ on $X$.

\item \textbf{Affine Scheme:}  
For a commutative ring $A$, the spectrum $\Spec A$ with the Zariski topology and the structure sheaf $\mathcal{O}_{\Spec A}$ is a locally ringed space. Each stalk $\mathcal{O}_{\mathrm{Spec}(A), \mathfrak{p}}$ is the localization $A_{\mathfrak{p}}$, which is a local ring. This locally ringed space is called \emph{affine scheme}, we also use $\Spec A$ to refer the affine scheme. 

For rings $A$ and $B$ and a ring homomorphism $\varphi : A \to B$, we get a morphism of locally ringed spaces:
\[
(f,f^\#):(\mathrm{Spec}(B), \mathcal{O}_{\mathrm{Spec}(B)}) \longrightarrow (\mathrm{Spec}(A), \mathcal{O}_{\mathrm{Spec}(A)})
\]
given by:
\begin{itemize}
\item Continuous map: $f : \mathrm{Spec}(B) \to \mathrm{Spec}(A)$, $f(\mathfrak{q}) = \varphi^{-1}(\mathfrak{q})$.
\item Sheaf morphism: $f^\#$ induced by $\varphi$ on localizations.
\end{itemize}

\item \textbf{Point and Double point:}
  We study more examples for affine scheme: $\Spec \bC$ and $D=\Spec \bC[x]/(x^2)$, they have the same underlying topological space, one point. But they have different structure sheaves:$ \bC $ and $ \bC[x]/(x^2) $. Image $D$ is the limit of two points $ \Spec \bC \sqcup \Spec \bC \cong \Spec \bC[x]/(x(x-c))$ collide together. 
\end{enumerate}
\end{example}

Moreover, one can show the realization functors $ R_{sm}: \Cart \to LRS, R_{alg}: \Ring^\op \to LRS $ are fully faithful, which means they extend to a functor on sheaves, in particular a fully faithful functor on locally representable sheaves:

\[
  R_{sm}: \Sch(\Cart) \to LRS, R_{alg}: \Sch(\Ring^\op) \to LRS
\]

Therefore, we can define smooth manifold (scheme) as a locally ringed space which locally isomorphic to Cartesian spaces (affine scheme).

\section{Geometry of Affine Scheme}

The idea of spectrum is to think the ring $A$ as functor ring on $\Spec A$: for $f \in A$ and point $ \mathfrak{p}\in \Spec A$, we should think $f(\mathfrak{p})= ev(f) \in A/\mathfrak{p} $, where $ev:A\to A/\mathfrak{p}$ is the quotient. 

Notice that the spectrum of ring form an unusual space, it is only $T_0$. So not all points are closed. It's helpful to think in following ways: It contains closed points, which is usual points; then for each irreducible algebraic curve $C$, we have a corresponding generic point $\xi_C$. Then for every closed point $x$ on that curve, $\xi_C$ is in the neighborhood of $x$, i.e. $x\in \overline{\{\xi_C\}}$, we denote this as $\xi_C \leadsto x $; Similarly, for each irreducible surface $S$, we have a generic point $\xi_S$, such that for the curve it contains, we have $\xi_S \leadsto \xi_C$.


Another interesting fact is the spectrum is a filter limit of finite $T_0$ space: You can consider a finite stratification of space by points, curve, surface, etc., then take the limit by have more and more of them. This is called \emph{Spectral space}.


We have the following diagram of corresponding geometric and algebraic objects.
\begin{center}
\begin{tabular}{|l|l|p{5cm}|}
\hline
\textbf{Geometry} & \textbf{Algebra} & \textbf{Connection} \\
\hline
Locally Ringed Space $(X,\cO_X)$ & Ring $A$ & $X=\Spec A$; $A=\cO_X(X)$. \\
\hline
Point $x \in X$ & Prime ideal $\mathfrak{p} \subset A$ &  \\
\hline
Closed point $x \in X$ & Maximal ideal $\mathfrak{m} \subset A$ &  \\
\hline
Function $f \in \cO_X(X)$, eval $f(x)$&  $ev_\mathfrak{p}:  A \to A/\mathfrak{p}$ & $f(\mathfrak{p}):= ev_\mathfrak{p}(f)\in A/\mathfrak{p} $  \\
\hline
Open subset $U \subseteq X$ & Localization $A_f$ & $U=D(f)=\Spec A_f=\{\mathfrak{p} \mid f(\mathfrak{p})\neq0 \Leftrightarrow f \notin\mathfrak{p} \}$. \\
\hline
Closed subset $Z \subseteq X$ & Ideal $I \subset A$ & $Z=V(I)=\Spec A/I =\{\mathfrak{p} \mid \forall f\in I, f(\mathfrak{p})=0 \Leftrightarrow I \subseteq \mathfrak{p} \}$; $I(Z)=\{f\in A \mid \forall \mathfrak{p} \in Z, f(\mathfrak{p})=0 \}= \bigcap_{ \mathfrak{p} \in Z } \mathfrak{p} $.  \\
\hline
Structure sheaf $\mathcal{O}_X$ &  $\Lim A_f$  & $\cO_{\Spec A}(D(f)) = A_f$ and $\cO_{\Spec A}(U) = \Lim_{D(f)\subseteq U} A_f$. \\
\hline
Stalk $\mathcal{O}_{X,x}$ & Local ring $A_{\mathfrak{p}}$ & $\mathcal{O}_{X,x}=A_{\mathfrak{p}}= (A\setminus \mathfrak{p})^{-1}A$. \\
\hline
Irreducible closed subset $F$ & Prime ideal $\mathfrak{p}$ & $F=\overline{\{\mathfrak{p}\}}$; generic point of $F$ is $\mathfrak{p}$. \\
\hline
Specialization $\xi \leadsto x$ ($x\in \overline{\{\xi\}} $) & $ \mathfrak{p}_{\xi}\subset \mathfrak{p}_x$ &\\
\hline
Morphism  $(X,\cO_X) \to (Y,\cO_Y)$ & Ring homomorphism $ B \to A$ &  $\varphi^{-1} : \Spec A \to \Spec B$ corresponds to $\varphi : B \to A$. \\
\hline
Fiber product $X\times_Z Y$ & Tensor product $A\otimes_C B$ & $\Spec (A\otimes_C B)= \Spec A \times_{\Spec C} \Spec B $\\
\hline
Fiber $f^{-1}(y)$ over $y \in Y$ & Tensor product $A \otimes_B \kappa(y)$ & $ f^{-1}(y)= \Spec( A \otimes_B \kappa(y))$ where residue field $\kappa(y)= \mathrm{Frac}(B/\mathfrak{p}_y)$. \\
\hline
Cotangent space $T^*_x X$ & $\kappa( \mathfrak{m})$-Vector space $\mathfrak{m}/\mathfrak{m}^2$&\\
\hline 
Quasi-Coherent sheaf $ \cF \in \Sh(X)$ & $A$-module $M$ & $\cF=\widetilde{M}$;$M=\cF(X)$\\
\hline 
Vector bundle $ E$ & Locally free module $M$ & $M=\Gamma(X,E)$\\
\hline 
Section $ \cF(U)$  & Localization $M_f$ & $\cF(D(f))= M_f=M\otimes_A A_f$ and $\cF(U) = \Lim_{D(f)\subseteq U} M_f$\\
\hline
\end{tabular}
\end{center}

\begin{example}
  \begin{itemize}
    \item \textbf{Spectrum of $\bC[x]$} 
Prime ideals:
\[
\mathrm{Spec}(\mathbb{C}[x]) = \{ (0) \} \cup \{ (x - a) \mid a \in \mathbb{C} \}.
\]
Generic point: $(0)$; closed points: $(x - a)$ for $a \in \mathbb{C}$.

\begin{center}
\begin{tikzpicture}[node distance=1.5cm]
\node (generic) [circle, draw] {$(0)$};
\node (a0) [left=of generic] {$\cdots$};
\node (a1) [below left=of generic] {$(x-a_1)$};
\node (a2) [below=of generic] {$(x-a_2)$};
\node (a3) [below right=of generic] {$(x-a_3)$};
\node (a4) [right=of generic] {$\cdots$};
\draw[->,decorate, decoration={snake, amplitude=.7mm, segment length=3mm}] (generic) -- (a1);
\draw[->,decorate, decoration={snake, amplitude=.7mm, segment length=3mm}] (generic) -- (a2);
\draw[->,decorate, decoration={snake, amplitude=.7mm, segment length=3mm}] (generic) -- (a3);
\node[below=2cm of generic] {$\text{Affine line over } \mathbb{C}$};
\end{tikzpicture}
\end{center}
    \item \textbf{Spectrum of $\bZ$} 
\[
\mathrm{Spec}(\mathbb{Z}) = \{ (0) \} \cup \{ (p) \mid p \text{ prime} \}.
\]
Generic point: $(0)$; closed points: $(p)$ for $p \in \mathbb{N}$ prime.
\begin{center}
\begin{tikzpicture}[node distance=1.5cm]
\node (genericZ) [circle, draw] {$(0)$};
\node (a0) [left=of genericZ] {$\cdots$};
\node (a4) [right=of genericZ] {$\cdots$};
\node (p2) [below left=of genericZ] {$(2)$};
\node (p3) [below=of genericZ] {$(3)$};
\node (p5) [below right=of genericZ] {$(5)$};
\draw[->,decorate, decoration={snake, amplitude=.7mm, segment length=3mm}] (genericZ) -- (p2);
\draw[->,decorate, decoration={snake, amplitude=.7mm, segment length=3mm}] (genericZ) -- (p3);
\draw[->,decorate, decoration={snake, amplitude=.7mm, segment length=3mm}] (genericZ) -- (p5);
\node[below=2cm of genericZ] {$\text{Points indexed by primes}$};
\end{tikzpicture}
\end{center}
    \item \textbf{Spectrum of $\bZ[x]$} 
Prime ideals:
\[
\begin{aligned}
&(0), \\
&(p), p \text{ prime} \\
&(f(x)) \text{ irreducible over } \mathbb{Q}, \\
&(p, f(x)) \text{ irreducible mod } p.
\end{aligned}
\]
  \end{itemize}
\begin{center}
\begin{tikzpicture}[node distance=1.5cm]
\node (genericZX) [double, double distance=1mm, circle, draw] {$(0)$};
\node (p) [right=of genericZX, circle, draw] {$(p)$};
\node (fx) [below=of genericZX, circle, draw] {$(f(x))$};
\node (pf) [below=of p] {$(p,f(x))$};
\node (a0) [left=of genericZX] {$\cdots$};
\node (a4) [right=of p] {$\cdots$};
\draw[->,decorate, decoration={snake, amplitude=.7mm, segment length=3mm}] (genericZX) -- (p);
\draw[->,decorate, decoration={snake, amplitude=.7mm, segment length=3mm}] (genericZX) -- (fx);
\draw[->,decorate, decoration={snake, amplitude=.7mm, segment length=3mm}] (p) -- (pf);
\draw[->,decorate, decoration={snake, amplitude=.7mm, segment length=3mm}] (fx) -- (pf);
\node[below=0.1mm of fx] {$\text{Affine line over } \mathrm{Spec}(\mathbb{Z})$};
\end{tikzpicture}
\end{center}
\end{example}
For affine scheme, we have an analogue result of Nullstellensatz, which is easier to prove that original one.
\begin{exercise}[Generalized Nullstellensatz]
Let $A$ be any ring.
  \begin{enumerate}
    \item Show that for non-nilpotent $f\notin \sqrt{(0)}$(i.e. $ \forall n, f^n \neq 0$), $A_f \neq 0$, therefore $\Spec A_f= D(f)\neq \emptyset$. As a consequence, there exist a prime ideal $ \mathfrak{p} \in D(f)$, i.e. $f\notin \mathfrak{p}$.
    \item Show that $\bigcap_{ \mathfrak{p} \in \Spec A} \mathfrak{p} \subseteq \sqrt{(0)}$.
    \item Show that for nilpotent $g \in  \sqrt{(0)}$, For all prime ideals $\mathfrak{p}$, $g\in \mathfrak{p} $.
    \item Show that $\bigcap_{ \mathfrak{p} \in \Spec A} \mathfrak{p} = \sqrt{(0)}$, and thus $I(V(J))=\{f\in A \mid \forall \mathfrak{p} \in V(J), f(\mathfrak{p})=0 \}= \bigcap_{ \mathfrak{p} \supseteq J} \mathfrak{p}  = \sqrt{J}$. (The original Nullstellensatz is $ \bigcap_{ \text{maximal ideal } \mathfrak{m} \supseteq J} \mathfrak{m}  = \sqrt{J} $, which is stronger that this.)
  \end{enumerate}
   
\end{exercise}

\section{Quasi-Coherent Sheaf and Vector Bundle}

Over a locally ringed space
\begin{definition}
Let $(X, \mathcal{O}_X)$ be a ringed space. A sheaf of $\mathcal{O}_X$-modules $\mathcal{F}$ on $X$ is called \emph{quasi-coherent} if for every point $x \in X$, there exists an open neighborhood $U$ of $x$ and an exact sequence of $\mathcal{O}_X|_U$-modules
\[
\mathcal{O}_X|_U^{\oplus I} \longrightarrow \mathcal{O}_X|_U^{\oplus J} \longrightarrow \mathcal{F}|_U \longrightarrow 0,
\]
where $I$ and $J$ are (possibly infinite) index sets.
\end{definition}

To understand what quasi-coherent sheaf looks like, we introduce the support of sheaf:
\begin{definition}
  Let $X$ be a topological space and $\mathcal{F}\in \Sh(X,\mathbf{Ab})$ a sheaf of Abelian group. The \emph{support} of $\mathcal{F}$ is the subset
\[
\operatorname{Supp}(\mathcal{F}) = \{ x \in X \mid \mathcal{F}_x \neq 0 \},
\]
where $\mathcal{F}_x$ denotes the stalk of $\mathcal{F}$ at $x$. It is a closed subset of $X$.
\end{definition}

We give some example of quasi-coherent sheaf on affine scheme.

\begin{example}[Quasi-Coherent Sheaf]
  
\end{example}
% chapter Realization (end)
