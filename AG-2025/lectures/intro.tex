\chapter{Introduction}\label{chap:introduction}


\section{What's Algebraic Geometry?}

Algebraic geometry is the study of geometry which locally are solutions of polynomials. There are two levels of study:

\begin{itemize}
  \item (Local) The study of geometry that solutions of polynomials have. This is not evident, since we are not just working over $\bR$ or $\bC$ but potentially all fields (rings). So we need to build a more essential geometry that associated with algebra (the spectrum), and then we can translate between algebra and geometry. This is study of affine variety (scheme).
  \item (Synthetic) Just like from opens of $\bR^n$ to the manifolds, we study the space that locally isomorphic to affine variety, and this can be done in several ways. This is study of variety (scheme).
\end{itemize}

This lecture is supposed to be part of an algebraic geometry course which will more focus on the second level. And this level of ideas are not rest only in algebraic geometry, the question of "How to construct and classify generalized spaces from certain building block" can be asked in all kinds of geometry. So this part can stand alone as "(post-modern\footnote{A well accepted definition is "after World War II"}) synthetic geometry".

\section{Essentialism vs. Structuralism}
Synthetic geometry is about formalization of geometry by axioms that directly speak about the fundamental concepts of geometry – such as points and lines – instead of about a backdrop for such objects – such as Cartesian spaces. In contract to analytic way of thinking every geometry object consists of points, in synthetic geometry lines and curves (or even strings) are also a priori entities. Instead, we focus on the \textbf{relationship} between those geometry objects, such as "a point is on a line", "two lines intersect". This is a structuralism point of view, we don't know what an object is but its relationship with others.

We are using \textbf{ category theory } and \textbf{ sheaf theory }to deal with the relationship between basic spaces, and moreover, if we think category as some sort of space, the "geometry of category" can reveal more structure about the geometry. This formalism of synthetic geometry is developed via study algebraic geometry, nowadays, depending on the input building blocks, it plays an important role in all kinds of geometry : differential geometry, topology, quantum field theory, etc.


\section{What's Geometry in Physics?}
Actually quantum physics are related both local and synthetic levels of algebraic geometry.
\begin{itemize}
  \item (Local) The idea of spectrum of commutative ring come from $C^*$ algebra, and also spectrum of operator, which are essential in quantum physics. The duality between algebra and geometry already appear in Heisenberg / Schrödinger pictures.
  \item (Synthetic) In quantum field theory we need to study with space of field configuration (history), it is infinite dimensional and weird behaved, but we want to view it as a "manifold", to define metric, (path) integration, differential form on it. Synthetic geometry can be used to define those mathematically (i.e. smooth set). Nevertheless, gauge theory, super geometry, etc., can be put into the same framework \cite{nlab:geometry_of_physics}.
\end{itemize}
In more concrete case, we have the mirror symmetry which create a link between Gromov-Witten invariant and Hodge theory.
\section{Plan of Course}
We will focus on three sort of geometry and study they comparatively with synthetic method. 
\begin{itemize}
  \item (Smooth Set/Manifold) The building blocks are opens of $\bR^n$, very geometric examples, can be used as bridge toward usual geometry.
  \item (Simplical Set) The building blocks are simplexes $ \Delta $, simplest abstract examples.
  \item (Scheme) The building blocks are spectrum of commutative rings, main objective of the course.
\end{itemize}
