\chapter{Sheaf}\label{chap:Sheaf} % (fold)

As we mentioned early, presheaf is analogue of ``linear functional", to get a category of generalized space, we need to impose the ``continuous" condition, And sheaf is such ``continuous" presheaf. 

In analysis, continuous means preserve the limit, i.e. $f(\lim x_i)=\lim f(x_i)$. So we should also define limit in category, in some sense it describes how to approximate an object by others.

Some reference can be found in \cite{maclane2012sheaves}. 
\section{Limit and Colimit}

Let's begin with the easiest example of analysis: the limit of an increasing sequence.

If we view the real number poset $(\bR, \leq)$ as a category. Then an increasing sequence is an order preserving map from $(\mathbb{N},\leq)$ to $\bR$. i.e a functorr
$a_{(-)}:\mathbb{N} \to \bR$. Let us unwind the definition of the limit $ \lim a_i$: it is the supremum of $\{a_i\}$, i.e
\[
 \forall b\in \bR, \forall i \in \mathbb{N}, a_i \leq b \Leftrightarrow \lim a_i \leq b
\]
Recall for the poset category, morphism $ \Hom_{\bR}(a,b)$ can be seen as the proofs of propsition $a\leq b$: if there a morphism, $a \leq b$ is true, otherwise the proofs is empty, it is false. So we can rewrite it as
\[
  \forall b\in \bR, \prod_{ i \in \mathbb{N}} \Hom_\bR(a_i , b) \cong \Hom_\bR(\lim a_i, b)
\]
So we can think the limit is formally defined as a pre(co)sheaf $b \mapsto \lim h^{a_i}(b)$, and then if we can find an object who represent this pre(co)sheaf as $h^{\lim a_i}$, the limit exists as this object.

Then next example we consider a functor $X_{(-)}: \mathbb{N} \to \Set$, we should intuitively think it limit is $\bigcup_{i\in \mathbb{N}} X_i$, in this case it is called \textbf{Colimit}. But if we compare to the previous formula, we just get an inclusion:
\[
  \forall A\in \Set, \prod_{ i \in \mathbb{N}} \Hom_\Set(X_i , A) \supseteq \Hom_\Set(\bigcup_{i\in \mathbb{N}} X_i, A)
\]

The reason for this is we also need to ask the morphisms $ f_i \in \Hom_\Set(X_i , A)$ compactible which the morphism from functor $X_{i\leq j}:X_i\to X_{j}$, that is to say $f_i = f_j \circ X_{i\leq j}$.
% https://q.uiver.app/#q=WzAsNSxbMSwwLCJYX2kiXSxbMiwwLCJYX3tpKzF9Il0sWzMsMCwiXFxjZG90cyJdLFsxLDEsIkEiXSxbMCwwLCJcXGNkb3RzIl0sWzAsMSwiWF97aVxcbGVxIGkrMX0iXSxbMSwyLCJYX3tpKzFcXGxlcSBpKzJ9Il0sWzAsMywiZl9pIiwxXSxbMSwzLCJmX3tpKzF9IiwxXSxbMiwzXSxbNCwwLCJYX3tpLTFcXGxlcSBpfSJdLFs0LDNdXQ==
\[\begin{tikzcd}[ampersand replacement=\&,cramped]
	\cdots \& {X_i} \& {X_{i+1}} \& \cdots \\
	\& A
	\arrow["{X_{i-1\leq i}}", from=1-1, to=1-2]
	\arrow[from=1-1, to=2-2]
	\arrow["{X_{i\leq i+1}}", from=1-2, to=1-3]
	\arrow["{f_i}"{description}, from=1-2, to=2-2]
	\arrow["{X_{i+1\leq i+2}}", from=1-3, to=1-4]
	\arrow["{f_{i+1}}"{description}, from=1-3, to=2-2]
	\arrow[from=1-4, to=2-2]
\end{tikzcd}\]
This motivates us to give the definition of Limit and Colimit:
\begin{definition}[Limit and Colimit of Set]
  Let $D : J \to \mathbf{Set}$ be a diagram(functor) of sets indexed by a small category $J$. The \emph{limit} of $D$, denoted $\Lim_J D$, is the subset of the product
\[
\prod_{j \in J} D(j)
\]
consisting of all families $(x_j)_{j \in J}$ such that for every morphism $f : i \to j$ in $J$, we have $D(f)(x_i) = x_j$.

The \emph{colimit} of $D$, denoted $\Colim_J D$, is the quotient of the disjoint union
\[
\bigsqcup_{j \in J} D(j)
\]
by the equivalence relation generated by $x \sim D(f)(x)$ for every morphism $f : i \to j$ in $J$ and every $x \in D(i)$.
\end{definition}
We will omit $J$ sometimes.
\begin{remark}
 The intuition of limit is gluing functions, of colimit is gluing space. Image there is a covering of spaces $\bigcup X_i\to X$, to gluing space we start from $\bigsqcup X_i$ then we identify the intersections $X_i\cap X_j\hookrightarrow X_i,X_j$; To gluing function on $C(X_i)$ we start with $\prod C(X_i)$ then we impose compactibility condition when restrict to $C(X_i\cap X_j)$.
\end{remark}
\begin{exercise}
  If we view  $i \mapsto \Hom_\Set(X_i , A)$ as functor $ h_A(X_{(-)}): \mathbb{N}^\op \to \Set $, Then we have the relation between limit and colimit:
\[
  \forall A\in \Set, \Lim_{i\in \mathbb{N}^\op} \Hom_\Set(X_i , A) \cong \Hom_\Set(\Colim_{i\in \mathbb{N}} X_i, A)
\]
Show that this is hold in general for all category $J$
\[
  \forall A\in \Set, \Lim_{j\in J^\op} \Hom_\Set(X_j , A) \cong \Hom_\Set(\Colim_{j\in J} X_j, A)
\]
\[
  \forall A\in \Set, \Lim_{j\in J} \Hom_\Set(A, X_j ) \cong \Hom_\Set(A, \Lim_{j\in J} X_j)
\]
\end{exercise}
\begin{exercise}
  Consider the category of functor $ \Fun(J,\Set)$ where the morphism is natural transformation. For $A\in \Set$ let $c(A) \in \Fun(J,\Set)$ be the const functor. Show that 
\[
  \Hom_{\Fun(J,\Set)}(X_{(-)},c(A)) \cong \Lim_{j\in J^\op} \Hom_\Set(X_j , A) 
\]
\[
  \Hom_{\Fun(J,\Set)}(c(A),X_{(-)}) \cong \Lim_{j\in J} \Hom_\Set(A, X_i ) 
\]
\end{exercise}
\begin{example}
  \begin{itemize}
    \item \textbf{Product and Coproduct:} \\
      Consider the set $J$ viewed as a discrete category  and no non-identity morphisms. A functor $D : J \to \mathbf{Set}$ is simply a family of sets $\{A_j\}_{j\in J}$. The limit of $D$ is the product set $\prod_{j\in J} A_j$ and the colimit is the coproduct (disjoint union) $\bigsqcup_{j\in J} A_j$.

    \item \textbf{Equalizer and Coequalizer:} \\
    Let $J$ be the category with two objects $\alpha,\beta$ and two parallel morphisms $f, g : \alpha \to \beta$. A functor $D : J \to \mathbf{Set}$ consists of sets $A, B$ and functions $f, g : A \to B$. The limit (equalizer) is:
    \[
      \Lim D = \mathrm{Eq}(f,g):= \{ a \in A \mid f(a) = g(a) \}.
    \]
    And the colimit (coequalizer) is the quotient set:
    \[
    \Colim D = \mathrm{Coeq}(f,g):=B / \sim
    \]
    where $b \sim b'$ if there exists $a \in A$ such that $f(a) = b$ and $g(a) = b'$.
    \item \textbf{Pullback and Pushout: } \\
    Let $J$ be the diagram $\alpha \xrightarrow{f} \gamma \xleftarrow{g} \beta$. A functor $D : J \to \mathbf{Set}$ assigns sets $X, Y, Z$ and functions $f : X \to Z$, $g : Y \to Z$. The limit (pullback) is:
    \[
    \Lim D = X \times_Z Y := \{ (x, y) \in X \times Y \mid f(x) = g(y) \}.
    \]
  And for $J^\op$ be the diagram $\alpha \xleftarrow{f} \gamma \xrightarrow{g} \beta$. A functor $D : J^\op \to \mathbf{Set}$ assigns sets $X, Y, Z$ and functions $f :  Z\to X$, $g : Z \to Y$. the colimit (pushout) is: 
  \[
    \Colim D = X \sqcup_Z Y := (X \sqcup Y) / \sim
    \]
    where $\sim$ is the equivalence relation generated by $f(z) \sim g(z)$ for all $z \in Z$.

   \end{itemize}
\end{example}
\begin{remark}
  Actually these are all essential limit and colimit: limit can be seen as equalizer of product and colimit can be seen as coequalizer of coproduct:

  Limit:
\begin{itemize}
    \item Consider the product $\prod_{j \in J} D(j)$.
    \item For each morphism $f : i \to j$ in $J$, define two morphisms:
    \begin{align*}
        \alpha, \beta : \prod_{j \in J} D(j) \to \prod_{f : i \to j} D(j)
    \end{align*}
    where $\alpha$ sends $(x_j)_{j \in J}$ to $(D(f)(x_i))_{f : i \to j}$ and $\beta$ sends $(x_j)_{j \in J}$ to $(x_j)_{f : i \to j}$.
    \item The limit $\Lim D$ is the equalizer of $\alpha$ and $\beta$:
    \[
    \Lim D = \mathrm{Eq}(\alpha, \beta).
    \]
\end{itemize}
Colimit:
\begin{itemize}
    \item Consider the coproduct $\coprod_{j \in J} D(j)$.
    \item For each morphism $f : i \to j$ in $J$, define two morphisms:
    \begin{align*}
        \alpha', \beta' : \bigsqcup_{f : i \to j} D(i) \to \bigsqcup_{j \in J} D(j)
    \end{align*}
    where $\alpha'$ sends $x \in D(i)$ (in the $f : i \to j$ summand) to $D(f)(x) \in D(j)$, and $\beta'$ sends $x \in D(i)$ to $x$ viewed in $D(i)$.
    \item The colimit $\Colim D$ is the coequalizer of $\alpha'$ and $\beta'$:
    \[
    \Colim D = \mathrm{Coeq}(\alpha', \beta').
    \]
\end{itemize}
\end{remark}

\begin{remark}[Homotopy (Co)limit]
  As we mentioned earlier, everything should be a priori a space, for (co)limit, it is called homotopy (co)limit. so even for sets as discrete spaces, the homotopy colimit can be non discrete (limit are still discrete) and the colimit we have here is just set of connected components.
  
  We give an example how homotopy colimit is like for coequalizer: given $f,g:A\to B$ of sets, we construct a space of graph $\mathbf{Coeq}(f,g)$, begin with elements $b \in B $ as points, then for every element $a\in A$, add a segment between $f(a)$ and $g(a)$. We can see $ \pi_0\mathbf{Coeq}(f,g)=\mathrm{Coeq}(f,g)$, and contain more information, for example $\pi_1\mathbf{Coeq}(f,g)$.
\end{remark}

To definite limit and colimit for general category, we can make use of morphism:
\begin{definition}[Limit and Colimit]
  Let $\mathcal{C}$ be a category, $J$ a small category, and $D : J \to \mathcal{C}$ a functor (called a diagram in $\mathcal{C}$). A \emph{limit} of $D$ is an object $\Lim_J D$ of $\mathcal{C}$ such that 
\[
  \forall A\in \cC, \Lim_{j\in J} \Hom_\cC(A, D(j) ) \cong \Hom_\cC(A, \Lim_{J} D)
\]
A \emph{colimit} of $D$ is an object $\Colim_J D$ of $\mathcal{C}$ such that
\[
  \forall A\in \cC, \Lim_{j\in J^\op} \Hom_\cC(D(j) , A) \cong \Hom_\cC(\Colim_{ J} D, A)
\]
\end{definition}
\begin{example}[Poset and Lattice]
 In a thin category come from poset $P$ we have the $\Lim p $ is the meet $\bigwedge p$, and the colimit $\Colim p$ is the joint $ \bigvee p$. Then the thin category admitted all limit and colimit is a complete lattice.
\end{example}
\begin{remark}[Adjoint Preserving (Co)limit]
  As we have aready seen a lot, functors $F:\cC \to \cD$ $G:\cD \to \cC$ are called \emph{adjunction}\footnote{We also say $F$ is left adjoint of $G$, $G$ is right adjoint of $F$.}, if we have the bijection for all $A\in \cC, B \in \cD$:
  \[
    \Hom_\cD(F(A),B)\cong \Hom_\cC(A,G(B))
  \]
  Then if (co)limit exist in those category, then by definition
  \begin{align*}
     \Hom_\cD(F(\Colim A_i),B) &\cong \Hom_\cC(\Colim A_i,G(B))\\
    \cong \Lim \Hom_\cC( A_i,G(B)) & \cong  \Lim \Hom_\cD( F(A_i),B)\cong   \Hom_\cD(\Colim F(A_i),B)
  \end{align*}
  By Yoneda lemma this means $F(\Colim A_i)\cong \Colim F(A_i)$. Similarly, we also have $G(\Lim B_i) \cong \Lim G( B_i)$.
\end{remark}

Thanks to Yoneda lemma, to understand stand (co)limit of general category, we only need to understand its presheaf, and thus limit of $\Set$.


Then we introduce an important kind of colimit, \emph{filtered colimit}. 
\begin{definition}[Filtered Category]
  A small category $J$ is called \emph{filtered} if:
  \begin{enumerate}
        \item It is non-empty.
              \item For every pair of objects $j_1, j_2 \in \mathrm{Ob}(J)$, there exists an object $k \in \mathrm{Ob}(J)$ and morphisms $j_1 \to k$ and $j_2 \to k$.
                    \item For every pair of parallel morphisms $f, g : j \to j'$ in $J$, there exists an object $k$ and a morphism $h : j' \to k$ such that $h \circ f = h \circ g$.
  \end{enumerate}
\end{definition} 
\begin{example}
 For the category of open set $\Op(X)$, both $\Op(X)$ and $\Op(X)^\op$ are filtered
\end{example}
\begin{exercise}
 Let $I$ be any finite category, i.e with finite objects and morphisms, then for any functor $F:I\to J$, Show that $J$ is filtered iff there exists a object $j_F \in J$ and a natural transformation $F\to c(j_F)$.
\end{exercise}
Intuitively, a filtered category allows us to ``coherently glue" data indexed by $J$. The colimit definite from $D: J\to \cC$ is called \emph{filtered colimit}. Filtered colimit can be defined by more explicit quotient instead of that of general colimit.
\begin{exercise}
  Show that for $J$ filtered $D:J\to\Set$, the filtered colimit $\Colim_J D = \left( \coprod_{j \in J} F(j)\right) / \sim$, where 
\[
    (x, j) \sim (y, k) \quad \text{if there exist morphisms } f : j \to l,\, g : k \to l \text{ in } J \text{ such that } D(f)(x) = D(g)(y).
    \]
\end{exercise}
  \begin{exercise}
   \begin{enumerate}
     \item Let $I,J$ be any small categories, consider $D: I \times J \to \cC$. Then show that: $\Lim_I\Lim_J D \cong \Lim_{I\times J} D \cong \Lim_J\Lim_I D$, same for the colimit. (You just need prove it for $\Set$.)
     \item Let $I$ be finite, and $J$ be filtered category, consider $D:I\times J \to \Set$. Then show that $\Lim_I\Colim_J D \cong \Colim_J\Lim_I D$. That is to say filtered colimit preserve finite limit.
   \end{enumerate} 
  \end{exercise}
Limit and Colimit are not necessarily always existing for all $D:J\to\cC$, but just like for a space $X$ we can define its completion $\hat{X}$ to make limit exist, we can define a (co)completion of a category $\cC$ to make (co)limit exists. An easy observation is limit and colimit are interchanged in $\cC$ and $\cC^\op$, so let us be focus on the case for colimit. Notice that limit and colimit are admitted for $\Set$, then so do presheaf category $\PSh(\cC)$

We first introduce the category of elements for a presheaf, which will be the diagram for a colim to assembly spaces.
\begin{definition}[Category of Elements]
  Let $\mathcal{C}$ be a category and let $\mathcal{F} : \mathcal{C}^{\mathrm{op}} \to \mathbf{Set}$ be a presheaf. The \emph{category of elements} of $\mathcal{F}$, denoted $\int_\cC \mathcal{F}$, is defined as follows:

\begin{itemize}
    \item \textbf{Objects:} Pairs $(C, x)$ where $C$ is an object of $\mathcal{C}$ and $x \in \mathcal{F}(C)$.
    \item \textbf{Morphisms:} A morphism $(C, x) \to (D, y)$ is a morphism $f : C \to D$ in $\mathcal{C}$ such that
    \[
    \mathcal{F}(f)(y) = x.
    \]
    (Note: since $\mathcal{F}$ is contravariant, the direction of $f$ is $C \to D$, but the induced map goes $\mathcal{F}(D) \to \mathcal{F}(C)$.)

    \item \textbf{Composition and identities:} Inherited from the category $\mathcal{C}$.
\end{itemize}
\end{definition}

\begin{example}[Category of Elements of Simplicial Set]
  Let $X: \Delta^\op\to \Set$ be a simplicial set, Then the category of elements $\int_\Delta X$ will be

\begin{itemize}
  \item \textbf{Objects:} $X_0$ collection of points;$X_1$ collection of segments;$X_2$ collection of triangles, ...
  \item \textbf{Morphisms:} A morphism $ x \in X_n \to y \in X_{n+1}$ if  $d_i(y)=x $  such that $x$ is the $i$-th face of $y$, etc.
\end{itemize}
\end{example}
Intuitively we should think $\int_\cC \cF$ give us the blueprint to reassembly. For any functor $R:\cC \to \cD$, we should think \[
  \int_{C\in\cC} \cF(C) \times R(C) \left(\text{or } \int_{\cC} \cF \times R \right) := \Colim_{(C,x)\in\int_\cC \cF} R(C) \in \cD
\] 
Is the assembly of $\cF$ inside $\cD$.
\begin{exercise}
  Take the Yoneda embedding $\yo: \cC \to \PSh(\cC)$, we have the $\int_{\cC} \cF \times \yo \in \PSh(\cC)$. 
  \begin{enumerate}
    \item Let $ |-|:\Delta \to \Top$ be defined as $ |[n]|= \Delta^n$, then we can extend it to all simplicial set $|-|:\PSh(\Delta)\to \Top$:
      \[
        |X|= \int_{[n]\in \Delta} X_n \times \Delta^n \in \Top
      \]
    Verify this is adjoint to $\mathrm{Sing}$:
  \[
    \Hom_{\Top}(|X|, T) \cong \Hom_{\PSh(\Delta)}(X, \mathrm{Sing}(T)).
  \]

    \item Show that for $B\in \cC$, we have a map 
      \[
      \cF(B) \to \int_{C\in\cC} \cF(C) \times \yo(C)(B)= \Colim_{(C,x)\in\int_\cC \cF} \Hom_\cC(B,C)
    \] 
    defined by
    \[
      b\in \cF(B) \mapsto ((B,b), \mathrm{id}_B) \in \Colim_{(C,x)\in\int_\cC \cF} \Hom_\cC(B,C)
    \]
    which is a bijection. (This is analogue to $ f(b) =\int_{c\in\bR} f(c)\delta(c-b)$)
  \item This extended to a natural transformation $\cF \to \int_{\cC} \cF \times \yo$ which is an isomorphism.
  \item Let $\cD$ be a cocomplete category(i.e. admitted all colimit), then there is a map between
    \[
      \Fun(\cC,\cD) \to \Fun^{\mathrm{cocont}}(\PSh(\cC),\cD)
    \]
    where $\Fun^{\mathrm{cocont}}$ means cocontinuous functor i.e. preserve colimit, the map is defined by 
    \[
      R \in \Fun(\cC,\cD) \mapsto \left( \cF \in \PSh(\cC) \mapsto  \int_{\cC} \cF \times R \in \cD\right)
    \]
    Show that this is a bijection. And in other words, $\yo:\cC\to\PSh(\cC)$ is the \emph{free cocompletion} of $\cC$.
  \end{enumerate}
\end{exercise}

Notice that even if $\cC$ is cocompletion, which dosen't means $ \cC \cong \PSh(\cC)$, because $\yo :\cC \cong \PSh(\cC) $ is not cocontinuous, i.e $\yo(\Colim C)\neq \Colim \yo( C)$ in general. This can also been seen from $ \cF \in \PSh$ is not continuous in general: \[
  \cF(\Colim C)=\Hom_{\PSh(\cC)}(\yo(\Colim C), \cF) \neq  \Hom_{\PSh(\cC)}(\Colim \yo(C), \cF)=\Lim \cF(C)
\] 
So if we impose some continuous condition, we can get subcategory of continuous presheaf, which are called sheaf, behaves more close to the test category $\cC$.

\begin{exercise}
  Show that for $B\in\cC$, $\yo(B)$ are continuous, i.e. $ \yo(B)(\Colim_{j\in J} C_j)\cong \Lim_{j\in J} \yo(B)(C_j)$. In others words $\yo(B)$ is a sheaf. 
\end{exercise}

\section{Sheaf and Coverage}

If we ask to presheaf be continuous for all colimit we will get nothing more that representable ones (this is an analogue to the Riesz representation theorem), instead we can restrict which kinds of colimit that it preserve. 

\begin{example}[Ind-Object]
  Let $\cC$ have all finite colimit (i.e for $C:J\to \cC$ with $J$ finite), Then the ind-object  is the presheaf that preserve finite colimit, i.e. $\cF(\Colim_J C)\cong \Lim_J \cF(C)$ for $J$ finite, this is also called \emph{left exact}. Let $\mathrm{ind}(\cC)\subset \PSh(\cC)$ be the subcategory of ind-object.
  \begin{exercise}
   \begin{enumerate}
    \item For any $J$ filtered and $D:J\to \cC$, show that $\Colim_J \yo(D) \in \PSh(\cC)$ is actually an ind-object. 
    \item* For any $ \cF \in \mathrm{ind}(\cC)$, show that the category of elements $\int_\cC \cF$ is filtered. This shows that $ \cF\cong \int_\cC \cF\yo$ is a filter colimit, and $\mathrm{ind}(\cC)$ is generated by representable presheaf with filtered colimit. 
   \end{enumerate} 
  \end{exercise}

\end{example}

In geometry setting, the colimit we want to preserve are come from coverage. We should first decide a class of open morphism in $\cC$, then we decide which collection of open morphism is a covering. Now for example consider a covering of $j_1: U_1 \hookrightarrow X \hookleftarrow U_2 :j_2$, intuitively, $X$ should identify to the pushout colimit $|U_{\bullet}|:=U_1 \sqcup_{U_1\cap U_2} U_2$. The presheaf preserve this colimit will satisfy
\[
  \cF(X) = \cF(U_1 \sqcup_{U_1\cap U_2} U_2) \cong \cF(U_1) \times_{\cF(U_1\cap U_2)}\cF(U_2)
\] 
The reason of consider this sheaf is because of locality, we want the generalized space $Y$ have can be tested locally, that is 
\[ \Map(X,Y) \cong \Map(U_1,Y)\times_{\Map(U_1\cap U_2,Y)}\Map(U_2,Y)=\left\{\substack{\text{maps from $U_1,U_2$ to $Y$} \\ \text{that coincide on $U_1\cap U_2$}}\right\}\]

Notice that the gluing $|U_{\bullet}|:=U_1 \sqcup_{U_1\cap U_2} U_2$ might not exists in $\cC$, but it always exists in $\PSh(\cC)$.

Let us definite it more formally 
\begin{definition}[Coverage(Grothendieck Topology)]
  Let $\mathcal{C}$ be a category. Given a class of \emph{basic open morphisms} of $\cC$, a \emph{coverage} $T$ on $\mathcal{C}$ assigns to each object $X \in \mathcal{C}$ a collection of families of basic open morphisms $\{ U_i \to X \}_{i \in I}$, called \emph{covering families}, satisfying the following condition:

 If $\{ U_i \to X \}_{i \in I}$ is a covering family and $f : Y \to X$ is any morphism in $\mathcal{C}$, then there exists a covering family $\{ V_j \to Y \}_{j \in J}$ such that for each $j$, the morphism $V_j \to Y \xrightarrow{f} X$ factors through some $U_i \to X$.

 A category $\cC$ equip with a coverage $T$ is call a \emph{site} $(\cC,T)$. Given a covering families, $\{ U_i \to X \}_{i \in I}$, let $U_{ij}:= U_i \times_X U_j$, then the \emph{Cech Nerve} is the presheaf $|U_{\bullet}|:= \bigsqcup_{i\in I} \yo(U_i) / \sim_{\yo(U_{ij})} \in \PSh(\cC)$.
\end{definition}

With all these definition, we can finally define sheaf
\begin{definition}[Sheaf]
 Let $(\cC,T)$ be a site, then sheaf $\cF\in\Sh_T(\cC)\subset \PSh(\cC)$ is the presheaf preserve gluing, i.e. for any covering families, $\{ U_i \to X \}_{i \in I}$
   \[
     \cF(X)\cong \cF(|U_{\bullet}|) =\Hom_{\PSh(\cC)}(|U_{\bullet}|, \cF)= \Lim \Hom_{\PSh(\cC)}(U_{\bullet},\cF)=\Lim \cF(U_{\bullet})
   \] 
  Here \[
    \Lim \cF(U_{\bullet}) := \left\{ (f_i)\in \prod_{i\in I}\cF(U_{i}) \,\middle\vert\, f_i|_{U_{ij}}=f_j|_{U_{ij}}\in \cF(U_{ij})\right\}
  \] 
  And the morphisms of sheaves are morphisms of them as presheaves.
\end{definition}
When the coverage is clear, we can omit it to just write $ \Sh(\cC)$
We have a lot of examples.
\begin{example}[Sheaf on a Space]
  Let $X$ be a topological space, consider the usual coverage for $\Op(X)$: $\{ U_i \to U \}_{i \in I}$, is a covering family iff $ \bigcup_{i\in I} U_i =U$. Then we have the sheaf category $\Sh(X):= \Sh(\Op(X))$.
  
  \begin{exercise}
   \begin{enumerate}
    \item Show that the continuous functor presheaf $C(U)$ is a sheaf.
    \item Show that for a vector bundle, the presheaf of section $\Gamma_E$ is a sheaf.
   \end{enumerate} 
  \end{exercise}
\end{example}

\begin{example}[Smooth Set]
  Let us define the coverage on $\Cart$: The basic open morphism is just open embedding of balls, \emph{good open covers} is $\{ U_i \to \bR^n \}_{i \in I}$ such that for all finite subset $J\subset I $, the intersection $\bigcap_{i\in J}U_i \cong \bR^n$ are also open balls. We always consider the sheaf of this coverage
  \begin{exercise}
   \begin{enumerate}
    \item Show that good open covers gives a coverage. 
    \item Show that a manifold $M$ is a sheaf.
    \item Show that the differential form $\Omega^k$ is a sheaf (but not a manifold).
   \end{enumerate} 
  \end{exercise}
\end{example}

\begin{example}[Zariski Site]
  Let us define the \emph{Zariski Topology} on $\Ring^\op$: Then base open morphism is given by localization, for $ f\in R, R\to R_f $. Now a Zariski cover \emph{good open covers} is $\{ R\to R_{f_i} \}_{i \in I}$ such that $(f_i)_{i\in I}=1$, or more explicitly, there exist some finite subset $j\in J\subset I$ and $ a_j \in R$ such that $\sum_{j\in J} a_jf_j=1$
   \begin{enumerate}
    \item Show that Zariski covers gives a coverage. 
    \item Recall $ \mathbb{A}^1=U : \Ring \to \Set $ send the ring to its underlining set. Show that the functor $\mathbb{A}^1$ is a sheaf. 

   \end{enumerate} 
\end{example}

\begin{example}[Kan complex]
  
\end{example}
\section{Sheafification}

We have seen sheaf is a subcategory of presheaf $\iota : \Sh(\cC) \hookrightarrow \PSh(\cC)$, actually for any presheaf we can associate sheaf as the ``best approximate". That is the sheafification functor $L:\PSh(\cC) \to \Sh(\cC)$ such that we have the bijection
\[
  \Hom_{\PSh(\cC)}(\cF, \iota \cG) \cong \Hom_{\Sh(\cC)}(L\cF, \cG)
\]
\begin{remark}
  Take the analogue of distribution, Let $ P: D\to \bR$ be the space of all linear functional, and we have an inclusion $ \yo:D\to P, f\mapsto <-,f>$. Now consider the subspace $K=\mathrm{Span}<\lim \yo(x_i) - \yo(\lim x_i)>$ generated by all converge sequence $x_i$, then the continuous functional is just $K^\perp$:
\[
  \iota: D'=K^\perp=\{f\in P \mid \forall x_i, \lim<x_i,f>-<\lim x_i, f>=0\}  \hookrightarrow P
\]
On other hand we can understand $D' \cong P/K$ as a quotient. The is give the projective map $ L: P\to D' $ such that $ <f,\iota(g)>_P = <L(f),g>_{D'} $
\end{remark}

To definition the sheafification, we want to identify all Cech nerves $|U_{\bullet}|$ of a covering $\{ U_i \to X \}_{i \in I}$ with $X$, i.e. $L|U_{\bullet}|\cong L\yo X= \yo X$, in other words we want to invert the morphism $ |U_{\bullet}| \to \yo X$ in $\PSh(\cC)$. Therefore, abstractly, we have the Bousfield localization

$$L:\PSh(\cC)\to \PSh(\cC)[(|U_{\bullet}| \to \yo X)^{-1}] \cong \Sh(\cC) $$


To give more explicit construction of sheafification, let us consider again the simplest example: a covering of space $j_1: U_1 \hookrightarrow X \hookleftarrow U_2 :j_2$, if we compare $\yo X$ and  $|U_{\bullet}|:= \yo U_1 \sqcup_{\yo (U_1\cap U_2)} \yo U_2$, for any space $Y$, we have 

\[|U_{\bullet}|(Y)= \Map(Y,U_1) \sqcup_{\Map(Y,U_1\cap U_2)}\Map(Y,U_1) \subsetneq \Map(Y,X) = \yo X(Y)\]
This is not equal in general, since we only have the map land in $U_1$ or $U_2$. To solve this problem we consider a cover of $V_1 \hookrightarrow Y \hookleftarrow V_2$, we define
\begin{align*}
  &L_0|U_{\bullet}|(Y):=|U_{\bullet}|(V_1) \times_{|U_{\bullet}|(V_1\cap V_2)}|U_{\bullet}|(V_2)\\
  =& |U_{\bullet}|(Y)\sqcup (\Map(V_1,U_1)' \times_{\Map(V_1\cap V_2, U_1\cap U_2)}\Map(V_2,U_2)')\\
   &\sqcup (\Map(V_2,U_1)' \times_{\Map(V_1\cap V_2, U_1\cap U_2)}\Map(V_1,U_2)') / \sim \subsetneq \Map(Y,X)\\
   &\text{where } \Map(V_i,U_j)'=\{f\in \Map(V_i,U_j)\mid f(V_1\cap V_2)\subset U_1\cap U_2\}
\end{align*}
This is closer to $ \Map(Y,X)$, in fact for any $ f \in \Map(Y,X)$, we have the cover $f^{-1}(U_i)\hookrightarrow Y$ and $f \in \Map(f^{-1}(U_1),U_1)' \times_{\Map(f^{-1}(U_1\cap U_2), U_1\cap U_2)}\Map(f^{-1}(U_i),U_2)'$. That is to say if we take colimit all possible covering, we can finally recover $\Map(Y,X)$ :
\[
  L|U_{\bullet}|(Y):=\Colim_{\{V_i\hookrightarrow Y\}}|U_{\bullet}|(V_1) \times_{|U_{\bullet}|(V_1\cap V_2)}|U_{\bullet}|(V_2) \cong \Map(Y,X)
\]
This motivates us to give following definitions
\begin{definition}[+ construction]
  
\end{definition}

\begin{remark}
  We can see the + construction is a filter colimit, thus it commute with finite limit. As a consequce, functor $(-)^+$ and $L$ is left exact, i.e. preserving finite limit.
\end{remark}

In the case of sheaf $\Sh(X)$ on a topological space $X$, we have even more concrete construction: the idea is every sheaf is a sheaf of sections. 

\begin{definition}[Etale Space]
  
\end{definition}

\begin{exercise}[(Co)Limit of Sheaf and Presheaf]
  Recall the (co)limit of presheaf is taking objectwise, i.e. $(\Lim_{\PSh} \cF_i)(X) =\Lim_{\PSh} \cF_i(X), (\Colim_{\PSh} \cF_i)(X) =\Colim_{\PSh} \cF_i(X)$. Let $\cF_{-}:I \to \Sh(\cC)$ be some diagram of sheaves.
 \begin{enumerate}
   \item Show that the presheaf limit $ \Lim_{\PSh} \iota \cF_i$ is still a sheaf. Therefore $ \iota(\Lim_{\Sh} \cF_i) \cong \Lim_{\PSh} \iota \cF_i$ we can compute sheaf limit by presheaf limit.
   \item Use the example above to show $   \iota(\Colim_{\Sh} \cF_i) \not\cong \Colim_{\PSh} \iota \cF_i$ in general, but rather we have $ \Lim_{\Sh} \cF_i \cong L(\Lim_{\PSh} \iota \cF_i) $, that is to say the colimit of sheaves is the sheafification of its presheaves colimit.
 \end{enumerate} 
 We usually omit $ \iota$ and think $\Sh(\cC) \subset \PSh(\cC)$ as a subcategory.
\end{exercise}
  Now we can define several natural things. Recall the open subset is define by the union of open base, here we replace union by the sheaf colimit 

\begin{definition}[Open Morphism]
  For $X\in \cC$ and $\mathcal{U}\in \Sh(\cC)$, a morphism $j: \mathcal{U} \to \yo X$ is open if there is  family of basic open morphism $\{ U_i \to \mathcal{U} \to X \}_{i \in I}$ (not necessarily be a covering) factor through $ \mathcal{U}$ such that $ L|U_{\bullet}| \cong  \mathcal{U}$. 

  More generally, a morphism of sheaf $j:\cF \to \cG $ is open if for all $X \in \cC, f\in \cG(X) \cong \Hom_{\Sh(\cC)}(\yo X, \cG) $ the pullback $ \cF\times_{\cG} \yo X \to \yo X$ are open.
\end{definition}
Here we can see a general way of definition in sheaf: define by pullback along all $X \in \cC, f\in \cG(X) \cong \Hom_{\Sh(\cC)}(\yo X, \cG) $.

\begin{definition}[Open Covering]
  For $X\in \cC$ and a family of open morphisms $\{\mathcal{U}_j \to \yo X\}_{j\in J}$ is an \emph{open covering} if there is cover family of basic open morphism $\{V_{ij} \to U_{ij} \to \mathcal{U}_j \to X \}_{i \in I}$ factor through some $\mathcal{U}_j$. 

  More generally, a family of open morphisms $\{\mathcal{U}_j \to \cF\}_{j\in J} $ is an open covering if for all $X \in \cC, f\in \cF(X) \cong \Hom_{\Sh(\cC)}(\yo X, \cF) $ the pullback $\{\mathcal{U}_j \times_{\cF} \yo X \to \yo X\}_{j\in J} $ are open coverings.
\end{definition}


\begin{remark}
  The advantage of considering sheaf rather than presheaf is we can reconstruct it via a relative small colimit along open covering. That is to say for an open covering $\{\mathcal{U}_i \to \cF\}_{i\in I} $, we can define the Cech nerve same as before: let $\mathcal{U}_{ij}:= \mathcal{U}_i \times_X \mathcal{U}_j$,  and the presheaf $|\mathcal{U}_{\bullet}|:= \bigsqcup_{i\in I} \mathcal{U}_i / \sim_{\mathcal{U}_{ij}} \in \PSh(\cC)$, the \emph{Cech Nerve} is the sheafification $ L|\mathcal{U}_{\bullet}|$. We can show that $ L|\mathcal{U}_{\bullet}| \cong \cF$ in $\Sh(\cC$. If the open covering is giving by the representable objects $\yo U_i$, this is even simpler.
\end{remark}
This motivates us to give following definition
\begin{definition}[Locally Representable Sheaves]
  A sheaf $\cF $ is locally reprensentable if there is an open covering of representable sheaves $\{\yo U_i \to \cF\}_{i\in I} $. In other words we can think $\cF \cong L|U_{\bullet}|$ is gluing by test spaces $U_i \in \cC$

  In smooth set $\Sh(\Cart)$ locally reprensentable sheaf are \emph{smooth manifold}. In algebraic set $\Sh_{\mathrm{Zar}}(\Ring^\op)$ locally reprensentable sheaf is \emph{scheme}.
\end{definition}




\begin{exercise}[$S^1$ as a gluing space]
  
\end{exercise}
The advantage of locally representable sheaves

\section{Geometric morphism of Topos}
slice category
\section{Moduli Sheaves/Spaces}
