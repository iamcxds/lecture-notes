\chapter{Sheaf}\label{chap:Sheaf} % (fold)

As we mentioned early, presheaf is analogue of "linear functional", to get a category of generalized space, we need to impose the "continuous" condition, And sheaf is such "continuous" presheaf. 

In analysis, continuous means preserve the limit, i.e. $f(\lim x_i)=\lim f(x_i)$. So we should also define limit in category, in some sense it describes how to approximate an object by others.

\section{Limit and Colimit}

Let's begin with the easiest example of analysis: the limit of an increasing sequence.

If we view the real number poset $(\bR, \leq)$ as a category. Then an increasing sequence is an order preserving map from $(\mathbb{N},\leq)$ to $\bR$. i.e a functorr
$a_{(-)}:\mathbb{N} \to \bR$. Let us unwind the definition of the limit $ \lim a_i$: it is the supremum of $\{a_i\}$, i.e
\[
 \forall b\in \bR, \forall i \in \mathbb{N}, a_i \leq b \Leftrightarrow \lim a_i \leq b
\]
Recall for the poset category, morphism $ \Hom_{\bR}(a,b)$ can be seen as the proofs of propsition $a\leq b$: if there a morphism, $a \leq b$ is true, otherwise the proofs is empty, it is false. So we can rewrite it as
\[
  \forall b\in \bR, \prod_{ i \in \mathbb{N}} \Hom_\bR(a_i , b) \cong \Hom_\bR(\lim a_i, b)
\]
So we can think the limit is formally defined as a pre(co)sheaf $b \mapsto \lim h^{a_i}(b)$, and then if we can find an object who represent this pre(co)sheaf as $h^{\lim a_i}$, the limit exists as this object.

Then next example we consider a functor $X_{(-)}: \mathbb{N} \to \Set$, we should intuitively think it limit is $\bigcup_{i\in \mathbb{N}} X_i$, in this case it is called \textbf{Colimit}. But if we compare to the previous formula, we just get an inclusion:
\[
  \forall A\in \Set, \prod_{ i \in \mathbb{N}} \Hom_\Set(X_i , A) \supseteq \Hom_\Set(\bigcup_{i\in \mathbb{N}} X_i, A)
\]

The reason for this is we also need to ask the morphisms $ f_i \in \Hom_\Set(X_i , A)$ compactible which the morphism from functor $X_{i\leq j}:X_i\to X_{j}$, that is to say $f_i = f_j \circ X_{i\leq j}$.
% https://q.uiver.app/#q=WzAsNSxbMSwwLCJYX2kiXSxbMiwwLCJYX3tpKzF9Il0sWzMsMCwiXFxjZG90cyJdLFsxLDEsIkEiXSxbMCwwLCJcXGNkb3RzIl0sWzAsMSwiWF97aVxcbGVxIGkrMX0iXSxbMSwyLCJYX3tpKzFcXGxlcSBpKzJ9Il0sWzAsMywiZl9pIiwxXSxbMSwzLCJmX3tpKzF9IiwxXSxbMiwzXSxbNCwwLCJYX3tpLTFcXGxlcSBpfSJdLFs0LDNdXQ==
\[\begin{tikzcd}[ampersand replacement=\&,cramped]
	\cdots \& {X_i} \& {X_{i+1}} \& \cdots \\
	\& A
	\arrow["{X_{i-1\leq i}}", from=1-1, to=1-2]
	\arrow[from=1-1, to=2-2]
	\arrow["{X_{i\leq i+1}}", from=1-2, to=1-3]
	\arrow["{f_i}"{description}, from=1-2, to=2-2]
	\arrow["{X_{i+1\leq i+2}}", from=1-3, to=1-4]
	\arrow["{f_{i+1}}"{description}, from=1-3, to=2-2]
	\arrow[from=1-4, to=2-2]
\end{tikzcd}\]
This motivates us to give the definition of Limit and Colimit:
\begin{definition}[Limit and Colimit of Set]
  Let $D : J \to \mathbf{Set}$ be a diagram(functor) of sets indexed by a small category $J$. The \emph{limit} of $D$, denoted $\Lim_J D$, is the subset of the product
\[
\prod_{j \in J} D(j)
\]
consisting of all families $(x_j)_{j \in J}$ such that for every morphism $f : i \to j$ in $J$, we have $D(f)(x_i) = x_j$.

The \emph{colimit} of $D$, denoted $\Colim_J D$, is the quotient of the disjoint union
\[
\bigsqcup_{j \in J} D(j)
\]
by the equivalence relation generated by $x \sim D(f)(x)$ for every morphism $f : i \to j$ in $J$ and every $x \in D(i)$.
\end{definition}
We will omit $J$ sometimes.
\begin{exercise}
  If we view  $i \mapsto \Hom_\Set(X_i , A)$ as functor $ h_A(X_{(-)}): \mathbb{N}^\op \to \Set $, Then we have the relation between limit and colimit:
\[
  \forall A\in \Set, \Lim_{i\in \mathbb{N}^\op} \Hom_\Set(X_i , A) \cong \Hom_\Set(\Colim_{i\in \mathbb{N}} X_i, A)
\]
Show that this is hold in general for all category $J$
\[
  \forall A\in \Set, \Lim_{j\in J^\op} \Hom_\Set(X_j , A) \cong \Hom_\Set(\Colim_{j\in J} X_j, A)
\]
\[
  \forall A\in \Set, \Lim_{j\in J} \Hom_\Set(A, X_j ) \cong \Hom_\Set(A, \Lim_{j\in J} X_j)
\]
\end{exercise}
\begin{exercise}
  Consider the category of functor $ \Fun(J,\Set)$ where the morphism is natural transformation. For $A\in \Set$ let $c(A) \in \Fun(J,\Set)$ be the const functor. Show that 
\[
  \Hom_{\Fun(J,\Set)}(X_{(-)},c(A)) \cong \Lim_{j\in J^\op} \Hom_\Set(X_j , A) 
\]
\[
  \Hom_{\Fun(J,\Set)}(c(A),X_{(-)}) \cong \Lim_{j\in J} \Hom_\Set(A, X_i ) 
\]
\end{exercise}
\begin{example}
  \begin{itemize}
    \item \textbf{Product and Coproduct:} \\
      Consider the set $J$ viewed as a discrete category  and no non-identity morphisms. A functor $D : J \to \mathbf{Set}$ is simply a family of sets $\{A_j\}_{j\in J}$. The limit of $D$ is the product set $\prod_{j\in J} A_j$ and the colimit is the coproduct (disjoint union) $\bigsqcup_{j\in J} A_j$.

    \item \textbf{Equalizer and Coequalizer:} \\
    Let $J$ be the category with two objects $\alpha,\beta$ and two parallel morphisms $f, g : \alpha \to \beta$. A functor $D : J \to \mathbf{Set}$ consists of sets $A, B$ and functions $f, g : A \to B$. The limit (equalizer) is:
    \[
      \Lim D = \mathrm{Eq}(f,g):= \{ a \in A \mid f(a) = g(a) \}.
    \]
    And the colimit (coequalizer) is the quotient set:
    \[
    \Colim D = \mathrm{Coeq}(f,g):=B / \sim
    \]
    where $b \sim b'$ if there exists $a \in A$ such that $f(a) = b$ and $g(a) = b'$.
    \item \textbf{Pullback and Pushout: } \\
    Let $J$ be the diagram $\alpha \xrightarrow{f} \gamma \xleftarrow{g} \beta$. A functor $D : J \to \mathbf{Set}$ assigns sets $X, Y, Z$ and functions $f : X \to Z$, $g : Y \to Z$. The limit (pullback) is:
    \[
    \Lim D = X \times_Z Y := \{ (x, y) \in X \times Y \mid f(x) = g(y) \}.
    \]
  And the colimit (pushout) is:
    \[
    \Colim D = X \sqcup_Z Y := (X \sqcup Y) / \sim
    \]
    where $\sim$ is the equivalence relation generated by $f(z) \sim g(z)$ for all $z \in Z$.

   \end{itemize}
\end{example}
\begin{remark}
  Anctually these are all essential limit and colimit: limit can be seen as equalizer of product and colimit can be seen as coequalizer of coproduct:

  Limit:
\begin{itemize}
    \item Consider the product $\prod_{j \in J} D(j)$.
    \item For each morphism $f : i \to j$ in $J$, define two morphisms:
    \begin{align*}
        \alpha, \beta : \prod_{j \in J} D(j) \to \prod_{f : i \to j} D(j)
    \end{align*}
    where $\alpha$ sends $(x_j)_{j \in J}$ to $(D(f)(x_i))_{f : i \to j}$ and $\beta$ sends $(x_j)_{j \in J}$ to $(x_j)_{f : i \to j}$.
    \item The limit $\Lim D$ is the equalizer of $\alpha$ and $\beta$:
    \[
    \Lim D = \mathrm{Eq}(\alpha, \beta).
    \]
\end{itemize}
Colimit
\begin{itemize}
    \item Consider the coproduct $\coprod_{j \in J} D(j)$.
    \item For each morphism $f : i \to j$ in $J$, define two morphisms:
    \begin{align*}
        \alpha', \beta' : \bigsqcup_{f : i \to j} D(i) \to \bigsqcup_{j \in J} D(j)
    \end{align*}
    where $\alpha'$ sends $x \in D(i)$ (in the $f : i \to j$ summand) to $D(f)(x) \in D(j)$, and $\beta'$ sends $x \in D(i)$ to $x$ viewed in $D(i)$.
    \item The colimit $\Colim D$ is the coequalizer of $\alpha'$ and $\beta'$:
    \[
    \Colim D = \mathrm{Coeq}(\alpha', \beta').
    \]
\end{itemize}
\end{remark}

To definite limit and colimit

