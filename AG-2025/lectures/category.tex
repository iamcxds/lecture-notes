\chapter{Category}\label{chap:category} 

There are many references available for category theory; in this course, we follow \cite{nlab:geometry_of_physics_--_categories_and_toposes}
\section{Definition and Examples}
\begin{definition}[Category]
A \textbf{category} $\mathcal{C}$ consists of:
\begin{itemize}
  \item A class of objects $\mathrm{Ob}(\mathcal{C})$.
    \item For each pair $A, B$, a set of morphisms $\mathrm{Hom}_{\mathcal{C}}(A, B)$.
    \item A composition operation $\circ$ of morphisms and identity morphisms $\mathrm{id}_A$ for each object $A$.
\end{itemize}

Subject to:
\begin{itemize}
    \item \textbf{Associativity:} $h \circ (g \circ f) = (h \circ g) \circ f$
    \item \textbf{Identity:} $\mathrm{id}_B \circ f = f = f \circ \mathrm{id}_A$
\end{itemize}
 
\end{definition}

For simplicity, we denote objects in a category $\mathcal{C}$ by $A \in \mathcal{C}$ and morphisms by $f: A \to B$. There are various ways to understand what a category is; let us explore this through examples:

\begin{example}[Concrete Category]
A concrete category can be viewed as a collection of mathematical structures, where morphisms are maps that preserve those structures.
\begin{itemize}
  \item \textbf{Set:} Objects are sets\footnote{Since the collection of all sets does not itself form a set, we refer instead to a \emph{class} of objects. However, if we restrict our attention to \emph{small} sets, then the collection of objects can be treated as a set. In this course, we will ignore set-theoretic subtleties and proceed informally.}; morphisms are functions between sets.
    \item \textbf{Grp:} Objects are groups; morphisms are group homomorphisms.
    \item \textbf{Vect:} Objects are vector spaces; morphisms are linear maps.
    \item \textbf{Top:} Objects are topological spaces; morphisms are continuous maps.
\end{itemize}
You can construct infinitely many examples from the mathematical structures you are familiar with.
\end{example}

At first glance, this abstraction may seem unnecessary. However, we can generalize familiar notions from set theory. For instance, consider a morphism $f: A \to B$ in a category $\mathcal{C}$:

\begin{itemize}
    \item \textbf{Isomorphism:} $f$ is an \emph{isomorphism} if there exists a morphism $g: B \to A$ such that:
    \[
      g \circ f = \mathrm{id}_A \quad \text{and} \quad f \circ g = \mathrm{id}_B (\text{i.e.} g=f^{-1})
    \]
    In this case, $A$ and $B$ are said to be \emph{isomorphic}.

    \item \textbf{Monomorphism:} $f$ is a \emph{monomorphism} (or \emph{mono}) if for all morphisms $g_1, g_2: X \to A$, we have:
    \[
    f \circ g_1 = f \circ g_2 \Rightarrow g_1 = g_2
    \]
    That is, $f$ is left-cancellable.

    \item \textbf{Epimorphism:} $f$ is an \emph{epimorphism} (or \emph{epi}) if for all morphisms $h_1, h_2: B \to Y$, we have:
    \[
    h_1 \circ f = h_2 \circ f \Rightarrow h_1 = h_2
    \]
    That is, $f$ is right-cancellable.

\end{itemize}
\begin{exercise}
Verify that in the category $\Set$, these definitions correspond to bijections, injections, and surjections, respectively.
\end{exercise}

Besides the concrete categories, in which object have its own meaning, we can have abstract categories than objects are meaningless without in the context of category.
\begin{example}[Classifying Category]
  A category can be viewed as an algebraic structure generalizing a monoid\footnote{An algebraic structure similar to a group, but without requiring inverses}. Indeed, for any object $A \in \mathcal{C}$, the set of endomorphisms $\mathrm{Hom}_{\mathcal{C}}(A, A)$ forms a monoid. Conversely, any monoid $M$ can be associated with a \textbf{classifying category} $BM$, which has a single object $\bullet$ and morphisms $\mathrm{Hom}_{BM}(\bullet, \bullet) = M$. In particular, for any group $G$, we obtain a \textbf{classifying space} $BG$, where all morphisms are isomorphisms.
\end{example}

We refer to $BG$ as a space because we can interpret its isomorphisms as paths in a certain topological space:

\begin{example}[Groupoid]
A \textbf{groupoid} is a category in which every morphism is an isomorphism. Given a groupoid $\mathcal{X}$, we can construct its \textbf{geometric realization} $|\mathcal{X}|$ as follows:
\begin{itemize}
    \item Take the objects $x \in \mathcal{X}$ as points.
    \item For each morphism $f: x \to y$, attach a segment from point $x$ to point $y$.
    \item For each relation $f \circ g = h$, attach a triangle with edges labeled by $f$, $g$, and $h$.
    \item Continue this process for higher-dimensional cells $\ldots$
\end{itemize}
We will formalize this construction when we introduce simplicial sets.

\begin{exercise}
Show that the set of connected components $\pi_0(|\mathcal{X}|, x)$ corresponds bijectively to the isomorphism class of $x$ in $\mathcal{X}$. Furthermore, observe that $\mathrm{Hom}_{\mathcal{X}}(x, x)$ is a group, and prove that the fundamental group $\pi_1(|\mathcal{X}|, x) \cong \mathrm{Hom}_{\mathcal{X}}(x, x)$.
\end{exercise}

Conversely, for a topological space $S$, we can define the \textbf{fundamental groupoid} $\Pi_1(S)$, where:
\begin{itemize}
    \item Objects are points of $S$.
    \item Morphisms $\mathrm{Hom}_{\Pi_1(S)}(a, b)$ are homotopy classes of paths from $a$ to $b$.
\end{itemize}

\begin{exercise}
Describe the composition law in $\Pi_1(S)$ and verify that it satisfies the axioms of a groupoid.
\end{exercise}
\end{example}

The examples above illustrate that morphisms in a category can carry rich structure. On the other hand, if morphisms are trivial (i.e., at most one between any two objects), we obtain a partially ordered set:

\begin{example}[Poset]
Let $(P, \leq)$ be a partially ordered set. We can regard $P$ as a category as follows:
\begin{itemize}
    \item \textbf{Objects:} Elements of $P$.
    \item \textbf{Morphisms:} For $x, y \in P$, there exists a unique morphism $f: x \to y$ if and only if $x \leq y$.
    \item \textbf{Composition:} If $x \leq y$ and $y \leq z$, then $x \leq z$, so the morphism $x \to z$ is the composition of $x \to y$ and $y \to z$.
    \item \textbf{Identity:} For each $x \in P$, the identity morphism $\mathrm{id}_x: x \to x$ corresponds to the reflexivity $x \leq x$.
\end{itemize}

This category is called \emph{thin}, meaning there is at most one morphism between any two objects.

A frequently used example is the poset of open subsets of a topological space $S$, denoted $(\mathrm{Op}(S), \subseteq)$.
\end{example}

Note that a set $A$ can be viewed as a category in two distinct ways: either as a trivial groupoid or as a trivial poset. And a category can be viewed as a combination of this two case\footnote{It is useful to think category as an oriented graph}.
% https://q.uiver.app/#q=WzAsNixbMSwxLCJcXG1hdGhiZntHcm91cG9pZH0iXSxbMSwwLCJcXG1hdGhiZntDYXRlZ29yeX0iXSxbMiwxLCJcXG1hdGhiZntTZXR9Il0sWzIsMCwiXFxtYXRoYmZ7UG9zZXR9Il0sWzAsMCwiXFxtYXRoYmZ7TW9ub2lkfSJdLFswLDEsIlxcbWF0aGJme0dyb3VwfSJdLFswLDEsIiIsMCx7InN0eWxlIjp7InRhaWwiOnsibmFtZSI6Imhvb2siLCJzaWRlIjoidG9wIn19fV0sWzIsMCwiIiwwLHsic3R5bGUiOnsidGFpbCI6eyJuYW1lIjoiaG9vayIsInNpZGUiOiJib3R0b20ifX19XSxbMywxLCIiLDIseyJzdHlsZSI6eyJ0YWlsIjp7Im5hbWUiOiJob29rIiwic2lkZSI6ImJvdHRvbSJ9fX1dLFsyLDMsIiIsMix7InN0eWxlIjp7InRhaWwiOnsibmFtZSI6Imhvb2siLCJzaWRlIjoidG9wIn19fV0sWzQsMSwiIiwyLHsic3R5bGUiOnsidGFpbCI6eyJuYW1lIjoiaG9vayIsInNpZGUiOiJ0b3AifX19XSxbNSw0LCIiLDIseyJzdHlsZSI6eyJ0YWlsIjp7Im5hbWUiOiJob29rIiwic2lkZSI6InRvcCJ9fX1dLFs1LDAsIiIsMix7InN0eWxlIjp7InRhaWwiOnsibmFtZSI6Imhvb2siLCJzaWRlIjoidG9wIn19fV0sWzUsMSwiIiwyLHsic3R5bGUiOnsibmFtZSI6ImNvcm5lciJ9fV0sWzIsMSwiIiwyLHsic3R5bGUiOnsibmFtZSI6ImNvcm5lciJ9fV1d
\[\begin{tikzcd}[ampersand replacement=\&,cramped]
{\mathbf{Monoid}} \& {\mathbf{Category}} \& {\mathbf{Poset}} \\
{\mathbf{Group}} \& {\mathbf{Groupoid}} \& {\mathbf{Set}}
\arrow[hook, from=1-1, to=1-2]
\arrow[hook', from=1-3, to=1-2]
\arrow[hook, from=2-1, to=1-1]
\arrow["\lrcorner"{anchor=center, pos=0.125, rotate=90}, draw=none, from=2-1, to=1-2]
\arrow[hook, from=2-1, to=2-2]
\arrow[hook, from=2-2, to=1-2]
\arrow["\lrcorner"{anchor=center, pos=0.125, rotate=180}, draw=none, from=2-3, to=1-2]
\arrow[hook, from=2-3, to=1-3]
\arrow[hook', from=2-3, to=2-2]
\end{tikzcd}\]

\begin{remark}
In the definition of a category, if we allow the morphism set $\mathrm{Hom}_{\mathcal{C}}(A, B)$ to carry additional structure—such as an Abelian group, vector space, groupoid, topological space, or even another category—we obtain an \textbf{enriched category}. In fact, it is often more natural to think of $\mathrm{Hom}_{\mathcal{C}}(A, B)$ as the set of connected components of a space:
\[
\mathrm{Hom}_{\mathcal{C}}(A, B) \cong \pi_0 \mathrm{Map}_{\mathcal{C}}(A, B).
\]

This reflects a general principle of the \textbf{Univalence Foundation}: mathematical structures should be treated as spaces (or types) from the outset, and the classical set-theoretic version can be recovered by taking the set of connected components. We will explore how to identify and work with these underlying geometric structures later in the course.
\end{remark} 

\section{Functor}

To define a map between two categories, it is natural to require that such a map respects the structure of morphisms.

\begin{definition}[Functor]
Let $\mathcal{C}$ and $\mathcal{D}$ be categories. A \textbf{functor} $F: \mathcal{C} \to \mathcal{D}$ consists of:
\begin{itemize}
    \item A function that assigns to each object $A \in \mathcal{C}$ an object $F(A) \in \mathcal{D}$.
    \item A function that assigns to each morphism $f: A \to B$ in $\mathcal{C}$ a morphism $F(f): F(A) \to F(B)$ in $\mathcal{D}$.
\end{itemize}
such that:
\begin{itemize}
    \item \textbf{Identity preservation:} $F(\mathrm{id}_A) = \mathrm{id}_{F(A)}$.
    \item \textbf{Composition preservation:} For all $f: A \to B$ and $g: B \to C$ in $\mathcal{C}$,
    \[
    F(g \circ f) = F(g) \circ F(f).
    \]
\end{itemize}
\end{definition}

We denote the set (later, the category) of functors from $\mathcal{C}$ to $\mathcal{D}$ by $\mathrm{Fun}(\mathcal{C}, \mathcal{D})$.

\begin{example}[Functors Between Concrete Categories]
Functors between concrete categories respect the underlying structures:
\begin{itemize}
    \item \textbf{Forgetful Functor:}
    \[
    U: \mathbf{Grp} \to \mathbf{Set}
    \]
    assigns to each group its underlying set, and to each group homomorphism the same function viewed as a map of sets. Similar forgetful functors exist for $\mathbf{Vect}$, $\mathbf{Top}$, etc.

    \item \textbf{Free Functor:}
    \[
    \mathrm{Free}: \mathbf{Set} \to \mathbf{Grp}
    \]
    assigns to each set the free group it generates, and to each function between sets the induced group homomorphism.

    \begin{exercise}
    Verify that $\mathrm{Free}$ is indeed a functor. Then show:
    \[
    \mathrm{Hom}_{\mathbf{Grp}}(\mathrm{Free}(A), G) \cong \mathrm{Hom}_{\mathbf{Set}}(A, U(G)).
    \]
    How would you define $\mathrm{Free}$ for $\mathbf{Vect}$?
    \end{exercise}

    \item \textbf{Discrete Functor:}
    \[
    \mathrm{Disc}: \mathbf{Set} \to \mathbf{Top}
    \]
    assigns to each set the discrete topological space, and to each function the same map viewed as continuous.

    \item \textbf{Connected Components:}
    \[
    \pi_0: \mathbf{Top} \to \mathbf{Set}
    \]
    assigns to each topological space its set of connected components.

    \begin{exercise}
    Show:
    \[
    \mathrm{Hom}_{\mathbf{Top}}(\mathrm{Disc}(A), S) \cong \mathrm{Hom}_{\mathbf{Set}}(A, U(S)), \quad
    \mathrm{Hom}_{\mathbf{Top}}(S, \mathrm{Disc}(A)) \cong \mathrm{Hom}_{\mathbf{Set}}(\pi_0(S), A).
    \]
    \end{exercise}
\end{itemize}
\end{example}

\begin{exercise}
  \begin{enumerate}
    \item Let $G, H$ be groups (or more generally, monoids). Show that functors $F: BG \to BH$ correspond bijectively to group (monoid) homomorphisms $f: G \to H$.
    \item Show that a functor $F: BG\to \mathbf{Vect}$ correspond bijectively to representation of $G$.
  \end{enumerate}
\end{exercise}

\begin{exercise}
\begin{enumerate}
    \item Let $\mathcal{X}, \mathcal{Y}$ be groupoids. Show that a functor $F: \mathcal{X} \to \mathcal{Y}$ induces a continuous map between their geometric realizations:
    \[
    |F|: |\mathcal{X}| \to |\mathcal{Y}|.
    \]
    Let $\mathbf{Grpd}$ be the category of groupoids with functors as morphisms. Show that geometric realization defines a functor:
    \[
    |\cdot|: \mathbf{Grpd} \to \mathbf{Top}.
    \]
    \item Similarly, show that the fundamental groupoid construction defines a functor:
    \[
    \Pi_1: \mathbf{Top} \to \mathbf{Grpd}.
    \]
    \item* Show the adjunction:
    \[
    \mathrm{Hom}_{\mathbf{Top}}(S, |\mathcal{X}|) \cong \mathrm{Hom}_{\mathbf{Grpd}}(\Pi_1(S), \mathcal{X}).
    \]
\end{enumerate}
\end{exercise}

\begin{exercise}
Let $P, Q$ be posets. Show that a functor $F: P \to Q$ is equivalent to an order-preserving function.
\end{exercise}

\begin{example}[Ring–Space Correspondence]
Given a topological space $X$, its ring of real continuous functions $C(X)$ defines a contravariant functor. A map $p: X \to Y$ induces a pullback:
\[
p^*: C(Y) \to C(X), \quad f \mapsto f \circ p.
\]
To formalize this, we introduce the \textbf{opposite category} $\mathcal{C}^{\mathrm{op}}$, which has the same objects as $\mathcal{C}$ but reverses the direction of morphisms:
\[
\mathrm{Hom}_{\mathcal{C}^{\mathrm{op}}}(A, B) = \mathrm{Hom}_{\mathcal{C}}(B, A).
\]
Then,
\[
C(-): \mathbf{Top}^{\mathrm{op}} \to \mathbf{Ring}
\]
is a functor.
\end{example}

\begin{remark}
A central theme in algebraic geometry is reversing this functor: constructing a space from a commutative ring. That is, defining a functor:
\[
\mathrm{Spec}: \mathbf{Ring}^{\mathrm{op}} \to \mathbf{Top},
\]
such that we have \emph{adjunction}:
\[
\mathrm{Hom}_{\mathbf{Ring}}(R, C(X)) \cong \mathrm{Hom}_{\mathbf{Top}}(X, \mathrm{Spec}(R)).
\]

\begin{exercise}*
Show that the underlying set of $\mathrm{Spec}(R)$ is:
\[
U(\mathrm{Spec}(R)) = \mathrm{Hom}_{\mathbf{Ring}}(R, \mathbb{R} ).
\]
What topology should be given to $\mathrm{Spec}(R)$?
\end{exercise}

In practice, we impose additional structure to make this correspondence well-behaved:
\begin{itemize}
    \item Between $C^*$-algebras and locally compact Hausdorff spaces, via the Gelfand representation theorem—where the term “spectrum” originates.
    \item Between commutative rings and locally ringed spaces, which is foundational in algebraic geometry.
\end{itemize}
\end{remark}

\begin{example}[Hom Functor]
For a fixed object $A$ in a category $\mathcal{C}$, the \textbf{Hom functor} is defined as:
\[
h^A := \mathrm{Hom}_{\mathcal{C}}(A, -): \mathcal{C} \to \mathbf{Set},
\]
which assigns to each object $B$ the set $\mathrm{Hom}_{\mathcal{C}}(A, B)$, and to each morphism $f: B \to C$ the function:
\[
h^A(f): \mathrm{Hom}_{\mathcal{C}}(A, B) \to \mathrm{Hom}_{\mathcal{C}}(A, C), \quad g \mapsto f \circ g.
\]
Similarly, we define the contravariant version:
\[
h_A := \mathrm{Hom}_{\mathcal{C}}(-, A): \mathcal{C}^{\mathrm{op}} \to \mathbf{Set}.
\]
\end{example}

\begin{exercise}
Show that:
\begin{itemize}
    \item $f$ is a monomorphism if and only if $h^A(f)$ is injective for all $A$.
    \item $f$ is an epimorphism if and only if $h_A(f)$ is injective for all $A$.
    \item $f$ is an isomorphism if and only if both $h^A(f)$ and $h_A(f)$ are bijective for all $A$.
\end{itemize}
\end{exercise}

\begin{remark}
We can think of objects in $\mathcal{C}$ as “test objects,” and a functor as a way of encoding data about how these tests behave. For example, in a sigma model with target space $M$, let $\mathcal{C}$ be the category of space-times. For each space-time $\Sigma$, the collection of fields $\mathrm{Map}(\Sigma, M)$ defines such a functor. The fundamental question is: given such a functor, can we reconstruct the underlying “space”?
\end{remark}

\section{Presheaf}

We now introduce a central concept for understanding "generalized spaces" in this course.

\begin{definition}[Presheaf]
Let $\mathcal{C}$ be a category. A \textbf{presheaf} on $\mathcal{C}$ is a functor:
\[
F: \mathcal{C}^{\mathrm{op}} \to \mathbf{Set}
\]
That is, $F$ assigns:
\begin{itemize}
    \item To each object $U$ in $\mathcal{C}$, a set $F(U)$.
    \item To each morphism $f: V \to U$ in $\mathcal{C}$, a function:
    \[
    F(f): F(U) \to F(V)
    \]
\end{itemize}
such that:
\begin{itemize}
    \item $F(\mathrm{id}_U) = \mathrm{id}_{F(U)}$
    \item $F(g \circ f) = F(f) \circ F(g)$ for composable morphisms $f$ and $g$ in $\mathcal{C}$
\end{itemize}
\end{definition}

Let $\PSh(\mathcal{C}) = \mathrm{Fun}(\mathcal{C}^{\mathrm{op}}, \mathbf{Set})$ denote the category of presheaves on $\mathcal{C}$.

\begin{remark}
To understand why presheaves represent generalized spaces, consider the analogy with distributions as generalized functions:
\begin{itemize}
    \item We begin with a space of test functions. Let $\mathcal{D}(\Omega)$ denote the space of smooth functions with compact support in $\Omega$. There is a natural pairing:
    \[
    \langle f, g \rangle := \int_\Omega f g
    \]
    This allows us to associate to each test function $f \in \mathcal{D}(\Omega)$ a continuous functional $T_f \in \mathcal{D}'(\Omega)$ defined by:
    \[
    T_f(\phi) := \int_\Omega f \phi.
    \]
    The space $\mathcal{D}'(\Omega)$ of distributions is thus a space of generalized functions.

    \item Similarly, starting from a category of test spaces $\mathcal{C}$, we consider the Hom functor:
    \[
    \mathrm{Hom}_{\mathcal{C}}(-, -): \mathcal{C}^{\mathrm{op}} \times \mathcal{C} \to \mathbf{Set}.
    \]
    For each object $A \in \mathcal{C}$, the presheaf $h_A := \mathrm{Hom}_{\mathcal{C}}(-, A)$ is \emph{representable}. Presheaves generalize these representable ones, just as distributions generalize smooth functions.

    \item In solving differential equations, we often first obtain a distributional solution, and then study its \emph{regularity}—how closely it resembles a smooth function locally—allowing us to conclude that the distribution is an actual function.

    \item In geometry, for example in moduli problems, we may first define a (pre)sheaf\footnote{sheaf to presheaf, is like continuous functional to functional, we ask some continuous condition on presheaf} and then study its \emph{representability}—whether it locally resembles a test space—allowing us to conclude that we have constructed an actual space.
\end{itemize}
\end{remark}

As we've seen, for any $A \in \mathcal{C}$, the presheaf $h_A$ is called representable. Just as there are distributions that are not smooth functions, there are presheaves that are not representable. Here are some concrete examples:

\begin{example}[Presheaf of Functions]
Let $X$ be a topological space. For each open set $U \in \mathrm{Op}(X)$, the set of continuous functions $C(U)$ defines a presheaf $C \in \PSh(\mathrm{Op}(X))$. For $U \subseteq V$, we have a restriction map $C(V) \to C(U)$. More generally, for any topological space $Y$, we can define a presheaf $C(-, Y) \in \PSh(\mathrm{Op}(X))$. Similarly, one can define presheaves of smooth, analytic, or locally constant functions.
\end{example}

\begin{example}[Presheaf of Sections]
Let $E \to X$ be a vector bundle over a topological space $X$. Define the presheaf of sections $\Gamma_E$ by assigning to each open set $U \subseteq X$ the set of continuous (or smooth) sections of $E$ over $U$:
\[
\Gamma_E(U) = \{ s : U \to E \mid s \text{ is a section of } E \text{ over } U \}.
\]
\end{example}

\begin{example}[Smooth Set]
Let $\Cart$ be the category of Cartesian spaces, with objects $\mathbb{R}^n$ and morphisms given by smooth maps. A presheaf on $\Cart$ is called a \textbf{smooth set}.
\begin{itemize}
  \item \textbf{Manifolds:} For any smooth manifold $M$, define a presheaf $M(\mathbb{R}^n) := \{f: \mathbb{R}^n\to M \mid f \text{ is smooth.}\}$. Thus, every manifold defines a smooth set.
  \item \textbf{Mapping Spaces:} For any smooth manifolds $M,N$, define a presheaf 
    $$\mathbf{C}^\infty(M,N)(\mathbb{R}^n) := \{f: M\times \mathbb{R}^n\to N \mid f \text{ is smooth.}\}.$$ 
    And we have $\Hom_{\PSh(\Cart)}(L,\mathbf{C}^\infty(M,N)) \cong \Hom_{\PSh(\Cart)}(L\times M, N) $.
  \item \textbf{Differential Forms:} Consider the presheaf $\Omega^k$ assigning to each $\mathbb{R}^n$ the space of $k$-forms $\Omega^k(\mathbb{R}^n)$, with pullbacks along smooth maps. Show that $\Omega^k$ is not representable by a manifold.
\end{itemize}
\end{example}

\begin{example}[Simplicial Set]

Let $\Delta$ be the category of simplices, whose objects are the $n$-simplices $[n]$, defined as the subset of $\mathbb{R}^{n+1}$:
\[
  [n] = \left\{ (t_0, t_1, \ldots, t_n) \in \mathbb{R}^{n+1} \;\middle|\; t_i \geq 0,\; \sum_{i=0}^n t_i = 1 \right\},
\]
with each vertex $(t_i = 1, t_{j \neq i} = 0)$ marked by $i$. The morphisms in $\Delta$ are linear maps that send vertices to vertices while preserving the order of the markings. For example, we have:
\begin{itemize}
  \item  Face maps $\delta_i : [n-1] \to [n]$, which are injections corresponding to the face given by $t_i = 0$. 
 \item Degeneracy maps $\sigma_i : [n+1] \to [n]$, which are surjections corresponding to projections of simplices,
\end{itemize}
for $0 \leq i \leq n$.

\begin{exercise}
  Show that $\Delta$ is equivalent to the category of finite totally ordered sets, whose objects are $[n] = \{ 0 < 1 < \cdots < n \}$, and whose morphisms are order-preserving maps.
\end{exercise}
A presheaf on $\Delta$ is called a \textbf{simplicial set}. More explicitly, a simplicial set $X$ is a sequence of sets $\{ X_n = X([n]) \}_{n \geq 0}$ together with:
\begin{itemize}
  \item Face maps $d_i = X(\delta_i) : X_n \to X_{n-1}$,
 \item Degeneracy maps $s_i = X(\sigma_i) : X_n \to X_{n+1}$,
\end{itemize}
for $0 \leq i \leq n$.

Intuitively, we can think of $X_n$ as encoding the data of a triangulated space: $X_0$ consists of points, $X_1$ of segments, $X_2$ of triangles, and so on. The face maps $d_i$ describe how these simplices are glued together via their boundaries.

\begin{itemize}
  \item Let $\Delta^n$ be the underlying topological space of $[n]$. For any topological space $X$, we can define a simplicial set $\mathrm{Sing}(X)$ where:
  \[
    \mathrm{Sing}(X)_n := \Hom_{\Top}(\Delta^n, X).
  \]

  \item For any small category $\cC$, we can define the \emph{nerve} simplicial set $N(\cC)$, where:
  \[
    N(\cC)_n := \Fun([n], \cC),
  \]
  viewing $[n]$ as a category via its poset structure.

  \item In fact, $\mathrm{Sing} : \Top \to \PSh(\Delta)$ is a functor. We aim to define its adjoint functor $|\cdot| : \PSh(\Delta) \to \Top$, called the \emph{geometric realization}, such that:
  \[
    \Hom_{\Top}(|S|, X) \cong \Hom_{\PSh(\Delta)}(S, \mathrm{Sing}(X)).
  \]
  Clearly, we have $|h_{[n]}| = \Delta^n$, and the intuition is that we glue together topological simplices according to the data encoded in the simplicial set. A formal definition will be given after we introduce colimits.
\end{itemize}
\end{example}

\begin{example}[Algebraic Set]
In algebraic geometry, we treat $\mathbf{Ring}^{\mathrm{op}}$ as the category of test spaces. A presheaf $\mathcal{F} \in \PSh(\mathbf{Ring}^{\mathrm{op}}) = \mathrm{Fun}(\mathbf{Ring}, \mathbf{Set})$ is then a functor from rings to sets, which we call an \textbf{algebraic set}.
\begin{itemize}
    \item \textbf{Affine Variety:} We begin with the geometry of zero set of polynomials. Let $P_1, \ldots, P_m \in \mathbb{Z}[x_1, \ldots, x_n]$ be polynomials. Define a functor $V_P: \mathbf{Ring} \to \mathbf{Set}$ by:
    \[
    V_P(R) = \{ (r_1, \ldots, r_n) \in R^n \mid P_i(r_1, \ldots, r_n) = 0 \text{ for all } i \}.
    \]
    \begin{exercise}
    Let $R_P = \mathbb{Z}[x_1, \ldots, x_n]/(P_1, \ldots, P_m)$. Show that $V_P = h^{R_P}$, i.e., $V_P(R) = \mathrm{Hom}_{\mathbf{Ring}}(R_P, R)$.
    \end{exercise}

    \item \textbf{Projective Space:} Then we consider a non-representable example. Define a functor $\mathbb{P}^n: \mathbf{Ring} \to \mathbf{Set}$ by:
    \[
    \mathbb{P}^n(R) = \{ (r_0, \ldots, r_n) \in R^{n+1} \mid \exists (u_0, \ldots, u_n) \in R^{n+1} \text{ s.t. } \sum u_i r_i = 1 \}/\sim,
    \]
    where $\sim$ identifies tuples under scalar multiplication by units in $R$.

    \begin{exercise}
    Show that this defines a functor. Compare this with the definition of projective space in differential geometry when $R = \mathbb{R}$.
    \end{exercise}
\end{itemize}
\end{example}

As we indicated earlier, $\PSh(\mathcal{C})$ should itself form a category. Let us now determine what the morphisms between presheaves ought to be.

Since representable presheaves $h_A$ and $h_B$ are thought of as generalized spaces, it is natural to expect that morphisms between them should reflect the morphisms in the original category $\mathcal{C}$. Indeed, given a morphism $f: A \to B$ in $\mathcal{C}$, we can define a map of sets:
\[
f \circ - : h_A(C) \to h_B(C), \quad \text{for each } C \in \mathcal{C},
\]
However, for this to define a morphism of presheaves, these maps must be compatible with the structure of the presheaves—i.e., they must commute with the restriction maps for all choices of $C$ and morphisms between them.

This leads us naturally to the definition of morphisms between presheaves as \emph{natural transformations}, which ensure such compatibility across the entire category.

\begin{definition}[Morphisms of Presheaves]
A \emph{morphism of presheaves} $\varphi: \mathcal{F} \to \mathcal{G}$ is a natural transformation between the functors $\mathcal{F}$ and $\mathcal{G}$. That is, for each object $U$ in $\mathcal{C}$, there is a function $\varphi_U: \mathcal{F}(U) \to \mathcal{G}(U)$ such that for every morphism $f: V \to U$ in $\mathcal{C}$, the following diagram commutes:
% https://q.uiver.app/#q=WzAsNCxbMCwwLCJcXGJ1bGxldCJdLFsxLDAsIlxcYnVsbGV0Il0sWzAsMSwiXFxidWxsZXQiXSxbMSwxLCJcXGJ1bGxldCJdLFswLDEsIlxcdmFycGhpX1UiXSxbMCwyLCJcXG1hdGhjYWx7Rn0oZikiLDJdLFsyLDMsIlxcdmFycGhpX1YiXSxbMSwzLCJcXG1hdGhjYWx7R30oZikiXV0=
\[\begin{tikzcd}[ampersand replacement=\&,cramped]
  \mathcal{F}(U) \&  \mathcal{G}(U) \\
\mathcal{F}(V)   \& \mathcal{G}(V)
	\arrow["{\varphi_U}", from=1-1, to=1-2]
	\arrow["{\mathcal{F}(f)}"', from=1-1, to=2-1]
	\arrow["{\mathcal{G}(f)}", from=1-2, to=2-2]
	\arrow["{\varphi_V}", from=2-1, to=2-2]
\end{tikzcd}\]

\end{definition}
\begin{exercise}
\begin{enumerate}
    \item Show that a morphism $f: A \to B$ in $\mathcal{C}$ induces a morphism of presheaves $\yo(f): h_A \to h_B$. This defines a functor:
    \[
    \yo: \mathcal{C} \to \PSh(\mathcal{C}), \quad A \mapsto h_A,
    \]
    called the \emph{Yoneda embedding}.
    \item* (Yoneda Lemma) Show that, for $A\in \cC$ and $F\in \PSh(\cC)$ there is a bijection:
    \[
     F(A) \cong \mathrm{Hom}_{\PSh(\mathcal{C})}(h_A, F).
    \]
    As a corollary, for any $A,B\in \cC$ there is a bijection of morphisms:
    \[
    \yo(-): \mathrm{Hom}_{\mathcal{C}}(A, B) \cong \mathrm{Hom}_{\PSh(\mathcal{C})}(h_A, h_B).
    \]
\end{enumerate}
\end{exercise}

This lemma is extremely important. It justifies embedding the category $\mathcal{C}$ into the category of presheaves $\PSh(\mathcal{C})$ in a way that fully preserves its morphism structure. In other words, $\mathcal{C}$ can be faithfully represented inside $\PSh(\mathcal{C})$ via the Yoneda embedding.
\begin{exercise}
  \begin{enumerate}
    \item 
Let $M$ be a smooth manifold viewed as a smooth set. Show that every differential form $\omega \in \Omega^k(M)$ induces a morphism of smooth sets:
\[
\omega: M \to \Omega^k.
\]
\item* Then prove that this defines a bijection:
\[
\Omega^k(M) \cong \mathrm{Hom}_{\mathrm{sm}}(M, \Omega^k).
\]
  \end{enumerate}
\end{exercise}

\begin{exercise}[Smooth Algebras]
Let $C^{\infty}\mathbf{Alg}$ be the category of product-preserving functors $A : \Cart \to \Set$.

\begin{enumerate}
  \item Show that $A(\bR)$ has a ring structure.

  \item Show that for any smooth manifold $M$, we can define a smooth algebra:
  \[
    C^\infty(M) : \bR^i \mapsto C^\infty(M, \bR^i).
  \]
  Thus, every manifold contravariantly defines a smooth algebra.

  \item We have a functor from smooth sets to smooth algebras:
  \[
    C^\infty : \PSh(\Cart) \to C^{\infty}\mathbf{Alg}^\op, \quad X \mapsto \left( \bR^i \mapsto \Hom_{\PSh(\Cart)}(X, \yo \bR^i) \right),
  \]
  and a functor from smooth algebras to smooth sets:
  \[
    \Spec : C^{\infty}\mathbf{Alg}^\op \to \PSh(\Cart), \quad A \mapsto \left( \bR^i \mapsto \Hom_{C^{\infty}\mathbf{Alg}}(A, C^\infty(\bR^i)) \right).
  \]
  Show that we have the adjunction:
  \[
    \Hom_{C^{\infty}\mathbf{Alg}}(A, C^\infty(X)) \cong \Hom_{\PSh(\Cart)}(X, \Spec(A)).
  \]

  \item Define the smooth algebra of \emph{dual numbers}:
  \[
    C^\infty(\mathbb{D}) = \bR[\epsilon]/(\epsilon^2) = \{ a + b\epsilon \mid \epsilon^2 = 0 \},
  \]
  which can be thought of as smooth functions on an \emph{infinitesimal interval}:
  \[
    \mathbb{D} = ``\{ x \in \bR \mid x^2 = 0 \}."
  \]
  Verify the functor structure of $C^\infty(\mathbb{D})$: For any smooth map $f : \bR^n \to \bR$, we have:
  \[
    C^\infty(\mathbb{D})(f) : C^\infty(\mathbb{D})^n \to C^\infty(\mathbb{D}), \quad (\cdots, a_i + b_i \epsilon, \cdots) \mapsto f(\cdots, a_i, \cdots) + \sum_{i=1}^n b_i \frac{\partial f}{\partial x_i} \epsilon.
  \]

  \item For any smooth set $X$, define the tangent bundle $TX$ to be the smooth set:
  \[
    TX(\bR^i) = \Hom_{C^{\infty}\mathbf{Alg}}(C^\infty(X), C^\infty(\mathbb{D}) \otimes_\bR C^\infty(\bR^i)).
  \]
  Compare this definition to the usual definition of the tangent bundle of a manifold.
\end{enumerate}
\end{exercise}
