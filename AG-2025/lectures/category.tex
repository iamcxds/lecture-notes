\chapter{Category}\label{chap:category} 

There are many references available for category theory; in this course, we follow \cite{nlab:geometry_of_physics_--_categories_and_toposes}
\section{Definition and Examples}
\begin{definition}[Category]
A \textbf{category} $\mathcal{C}$ consists of:
\begin{itemize}
  \item A class of objects $\mathrm{Ob}(\mathcal{C})$.
    \item For each pair $A, B$, a set of morphisms $\mathrm{Hom}_{\mathcal{C}}(A, B)$.
    \item A composition operation $\circ$ of morphisms and identity morphisms $\mathrm{id}_A$ for each object $A$.
\end{itemize}

Subject to:
\begin{itemize}
    \item \textbf{Associativity:} $h \circ (g \circ f) = (h \circ g) \circ f$
    \item \textbf{Identity:} $\mathrm{id}_B \circ f = f = f \circ \mathrm{id}_A$
\end{itemize}
 
\end{definition}

For simplicity, we denote objects in a category $\mathcal{C}$ by $A \in \mathcal{C}$ and morphisms by $f: A \to B$. There are various ways to understand what a category is; let us explore this through examples:

\begin{example}[Concrete Category]
A concrete category can be viewed as a collection of mathematical structures, where morphisms are maps that preserve those structures.
\begin{itemize}
  \item \textbf{Set:} Objects are sets\footnote{Since the collection of all sets does not itself form a set, we refer instead to a \emph{class} of objects. However, if we restrict our attention to \emph{small} sets, then the collection of objects can be treated as a set. In this course, we will ignore set-theoretic subtleties and proceed informally.}; morphisms are functions between sets.
    \item \textbf{Grp:} Objects are groups; morphisms are group homomorphisms.
    \item \textbf{Vect:} Objects are vector spaces; morphisms are linear maps.
    \item \textbf{Top:} Objects are topological spaces; morphisms are continuous maps.
\end{itemize}
You can construct infinitely many examples from the mathematical structures you are familiar with.
\end{example}

At first glance, this abstraction may seem unnecessary. However, we can generalize familiar notions from set theory. For instance, consider a morphism $f: A \to B$ in a category $\mathcal{C}$:

\begin{itemize}
    \item \textbf{Isomorphism:} $f$ is an \emph{isomorphism} if there exists a morphism $g: B \to A$ such that:
    \[
      g \circ f = \mathrm{id}_A \quad \text{and} \quad f \circ g = \mathrm{id}_B (\text{i.e.} g=f^{-1})
    \]
    In this case, $A$ and $B$ are said to be \emph{isomorphic}.

    \item \textbf{Monomorphism:} $f$ is a \emph{monomorphism} (or \emph{mono}) if for all morphisms $g_1, g_2: X \to A$, we have:
    \[
    f \circ g_1 = f \circ g_2 \Rightarrow g_1 = g_2
    \]
    That is, $f$ is left-cancellable.

    \item \textbf{Epimorphism:} $f$ is an \emph{epimorphism} (or \emph{epi}) if for all morphisms $h_1, h_2: B \to Y$, we have:
    \[
    h_1 \circ f = h_2 \circ f \Rightarrow h_1 = h_2
    \]
    That is, $f$ is right-cancellable.

\end{itemize}
\begin{exercise}
Verify that in the category $\Set$, these definitions correspond to bijections, injections, and surjections, respectively.
\end{exercise}

\begin{example}[Classifying Category]
  A category can be viewed as an algebraic structure generalizing a monoid\footnote{An algebraic structure similar to a group, but without requiring inverses}. Indeed, for any object $A \in \mathcal{C}$, the set of endomorphisms $\mathrm{Hom}_{\mathcal{C}}(A, A)$ forms a monoid. Conversely, any monoid $M$ can be associated with a \textbf{classifying category} $BM$, which has a single object $\bullet$ and morphisms $\mathrm{Hom}_{BM}(\bullet, \bullet) = M$. In particular, for any group $G$, we obtain a \textbf{classifying space} $BG$, where all morphisms are isomorphisms.
\end{example}

We refer to $BG$ as a space because we can interpret its isomorphisms as paths in a certain topological space:

\begin{example}[Groupoid]
A \textbf{groupoid} is a category in which every morphism is an isomorphism. Given a groupoid $\mathcal{X}$, we can construct its \textbf{geometric realization} $|\mathcal{X}|$ as follows:
\begin{itemize}
    \item Take the objects $x \in \mathcal{X}$ as points.
    \item For each morphism $f: x \to y$, attach a segment from point $x$ to point $y$.
    \item For each relation $f \circ g = h$, attach a triangle with edges labeled by $f$, $g$, and $h$.
    \item Continue this process for higher-dimensional cells $\ldots$
\end{itemize}
We will formalize this construction when we introduce simplicial sets.

\begin{exercise}
Show that the set of connected components $\pi_0(|\mathcal{X}|, x)$ corresponds bijectively to the isomorphism class of $x$ in $\mathcal{X}$. Furthermore, observe that $\mathrm{Hom}_{\mathcal{X}}(x, x)$ is a group, and prove that the fundamental group $\pi_1(|\mathcal{X}|, x) \cong \mathrm{Hom}_{\mathcal{X}}(x, x)$.
\end{exercise}

Conversely, for a topological space $S$, we can define the \textbf{fundamental groupoid} $\Pi_1(S)$, where:
\begin{itemize}
    \item Objects are points of $S$.
    \item Morphisms $\mathrm{Hom}_{\Pi_1(S)}(a, b)$ are homotopy classes of paths from $a$ to $b$.
\end{itemize}

\begin{exercise}
Describe the composition law in $\Pi_1(S)$ and verify that it satisfies the axioms of a groupoid.
\end{exercise}
\end{example}

The examples above illustrate that morphisms in a category can carry rich structure. On the other hand, if morphisms are trivial (i.e., at most one between any two objects), we obtain a partially ordered set:

\begin{example}[Poset]
Let $(P, \leq)$ be a partially ordered set. We can regard $P$ as a category as follows:
\begin{itemize}
    \item \textbf{Objects:} Elements of $P$.
    \item \textbf{Morphisms:} For $x, y \in P$, there exists a unique morphism $f: x \to y$ if and only if $x \leq y$.
    \item \textbf{Composition:} If $x \leq y$ and $y \leq z$, then $x \leq z$, so the morphism $x \to z$ is the composition of $x \to y$ and $y \to z$.
    \item \textbf{Identity:} For each $x \in P$, the identity morphism $\mathrm{id}_x: x \to x$ corresponds to the reflexivity $x \leq x$.
\end{itemize}

This category is called \emph{thin}, meaning there is at most one morphism between any two objects.

A frequently used example is the poset of open subsets of a topological space $S$, denoted $(\mathrm{Op}(S), \subseteq)$.
\end{example}

Note that a set $A$ can be viewed as a category in two distinct ways: either as a trivial groupoid or as a trivial poset. And a category can be viewed as a combination of this two case\footnote{It is useful to think category as an oriented graph}.
% https://q.uiver.app/#q=WzAsNCxbMSwyLCJTZXQiXSxbMCwxLCJHcm91cG9pZCJdLFsyLDEsIlBvc2V0Il0sWzEsMCwiQ2F0ZW9ncnkiXSxbMCwxLCIiLDAseyJzdHlsZSI6eyJ0YWlsIjp7Im5hbWUiOiJob29rIiwic2lkZSI6ImJvdHRvbSJ9fX1dLFswLDIsIiIsMix7InN0eWxlIjp7InRhaWwiOnsibmFtZSI6Imhvb2siLCJzaWRlIjoidG9wIn19fV0sWzEsMywiIiwwLHsic3R5bGUiOnsidGFpbCI6eyJuYW1lIjoiaG9vayIsInNpZGUiOiJ0b3AifX19XSxbMiwzLCIiLDIseyJzdHlsZSI6eyJ0YWlsIjp7Im5hbWUiOiJob29rIiwic2lkZSI6ImJvdHRvbSJ9fX1dXQ==
\[\begin{tikzcd}[ampersand replacement=\&,cramped]
\& \textbf{Category} \\
	\textbf{Groupoid} \&\& \textbf{Poset} \\
	\& \textbf{Set}
	\arrow[hook, from=2-1, to=1-2]
	\arrow[hook', from=2-3, to=1-2]
	\arrow[hook', from=3-2, to=2-1]
	\arrow[hook, from=3-2, to=2-3]
\end{tikzcd}\]

\begin{remark}
In the definition of a category, if we allow the morphism set $\mathrm{Hom}_{\mathcal{C}}(A, B)$ to carry additional structure—such as an Abelian group, vector space, groupoid, topological space, or even another category—we obtain an \textbf{enriched category}. In fact, it is often more natural to think of $\mathrm{Hom}_{\mathcal{C}}(A, B)$ as the set of connected components of a space:
\[
\mathrm{Hom}_{\mathcal{C}}(A, B) \cong \pi_0 \mathrm{Map}_{\mathcal{C}}(A, B).
\]

This reflects a general principle of the \textbf{Univalence Foundation}: mathematical structures should be treated as spaces (or types) from the outset, and the classical set-theoretic version can be recovered by taking the set of connected components. We will explore how to identify and work with these underlying geometric structures later in the course.
\end{remark} 

\section{Functor and Presheaf}
\subsection{Functor}

To define a map between two categories, it is natural to require that such a map respects the structure of morphisms.

\begin{definition}[Functor]
Let $\mathcal{C}$ and $\mathcal{D}$ be categories. A \textbf{functor} $F: \mathcal{C} \to \mathcal{D}$ consists of:
\begin{itemize}
    \item A function that assigns to each object $A \in \mathcal{C}$ an object $F(A) \in \mathcal{D}$.
    \item A function that assigns to each morphism $f: A \to B$ in $\mathcal{C}$ a morphism $F(f): F(A) \to F(B)$ in $\mathcal{D}$.
\end{itemize}
such that:
\begin{itemize}
    \item \textbf{Identity preservation:} $F(\mathrm{id}_A) = \mathrm{id}_{F(A)}$.
    \item \textbf{Composition preservation:} For all $f: A \to B$ and $g: B \to C$ in $\mathcal{C}$,
    \[
    F(g \circ f) = F(g) \circ F(f).
    \]
\end{itemize}
\end{definition}

We denote the set (later, the category) of functors from $\mathcal{C}$ to $\mathcal{D}$ by $\mathrm{Fun}(\mathcal{C}, \mathcal{D})$.

\begin{example}[Functors Between Concrete Categories]
Functors between concrete categories respect the underlying structures:
\begin{itemize}
    \item \textbf{Forgetful Functor:}
    \[
    U: \mathbf{Grp} \to \mathbf{Set}
    \]
    assigns to each group its underlying set, and to each group homomorphism the same function viewed as a map of sets. Similar forgetful functors exist for $\mathbf{Vect}$, $\mathbf{Top}$, etc.

    \item \textbf{Free Functor:}
    \[
    \mathrm{Free}: \mathbf{Set} \to \mathbf{Grp}
    \]
    assigns to each set the free group it generates, and to each function between sets the induced group homomorphism.

    \begin{exercise}
    Verify that $\mathrm{Free}$ is indeed a functor. Then show:
    \[
    \mathrm{Hom}_{\mathbf{Grp}}(\mathrm{Free}(A), G) \cong \mathrm{Hom}_{\mathbf{Set}}(A, U(G)).
    \]
    How would you define $\mathrm{Free}$ for $\mathbf{Vect}$?
    \end{exercise}

    \item \textbf{Discrete Functor:}
    \[
    \mathrm{Disc}: \mathbf{Set} \to \mathbf{Top}
    \]
    assigns to each set the discrete topological space, and to each function the same map viewed as continuous.

    \item \textbf{Connected Components:}
    \[
    \pi_0: \mathbf{Top} \to \mathbf{Set}
    \]
    assigns to each topological space its set of connected components.

    \begin{exercise}
    Show:
    \[
    \mathrm{Hom}_{\mathbf{Top}}(\mathrm{Disc}(A), S) \cong \mathrm{Hom}_{\mathbf{Set}}(A, U(S)), \quad
    \mathrm{Hom}_{\mathbf{Top}}(S, \mathrm{Disc}(A)) \cong \mathrm{Hom}_{\mathbf{Set}}(\pi_0(S), A).
    \]
    \end{exercise}
\end{itemize}
\end{example}

\begin{exercise}
Let $G, H$ be groups (or more generally, monoids). Show that functors $F: BG \to BH$ correspond bijectively to group (monoid) homomorphisms $f: G \to H$.
\end{exercise}

\begin{exercise}
\begin{enumerate}
    \item Let $\mathcal{X}, \mathcal{Y}$ be groupoids. Show that a functor $F: \mathcal{X} \to \mathcal{Y}$ induces a continuous map between their geometric realizations:
    \[
    |F|: |\mathcal{X}| \to |\mathcal{Y}|.
    \]
    Let $\mathbf{Grpd}$ be the category of groupoids with functors as morphisms. Show that geometric realization defines a functor:
    \[
    |\cdot|: \mathbf{Grpd} \to \mathbf{Top}.
    \]
    \item Similarly, show that the fundamental groupoid construction defines a functor:
    \[
    \Pi_1: \mathbf{Top} \to \mathbf{Grpd}.
    \]
    \item* Show the adjunction:
    \[
    \mathrm{Hom}_{\mathbf{Top}}(S, |\mathcal{X}|) \cong \mathrm{Hom}_{\mathbf{Grpd}}(\Pi_1(S), \mathcal{X}).
    \]
\end{enumerate}
\end{exercise}

\begin{exercise}
Let $P, Q$ be posets. Show that a functor $F: P \to Q$ is equivalent to an order-preserving function.
\end{exercise}

\begin{example}[Ring–Space Correspondence]
Given a topological space $X$, its ring of real continuous functions $C(X)$ defines a contravariant functor. A map $p: X \to Y$ induces a pullback:
\[
p^*: C(Y) \to C(X), \quad f \mapsto f \circ p.
\]
To formalize this, we introduce the \textbf{opposite category} $\mathcal{C}^{\mathrm{op}}$, which has the same objects as $\mathcal{C}$ but reverses the direction of morphisms:
\[
\mathrm{Hom}_{\mathcal{C}^{\mathrm{op}}}(A, B) = \mathrm{Hom}_{\mathcal{C}}(B, A).
\]
Then,
\[
C(-): \mathbf{Top}^{\mathrm{op}} \to \mathbf{Ring}
\]
is a functor.
\end{example}

\begin{remark}
A central theme in algebraic geometry is reversing this functor: constructing a space from a commutative ring. That is, defining a functor:
\[
\mathrm{Spec}: \mathbf{Ring}^{\mathrm{op}} \to \mathbf{Top},
\]
such that:
\[
\mathrm{Hom}_{\mathbf{Ring}}(R, C(X)) \cong \mathrm{Hom}_{\mathbf{Top}}(X, \mathrm{Spec}(R)).
\]

\begin{exercise}*
Show that the underlying set of $\mathrm{Spec}(R)$ is:
\[
U(\mathrm{Spec}(R)) = \mathrm{Hom}_{\mathbf{Ring}}(R, \mathbb{R} ).
\]
What topology should be given to $\mathrm{Spec}(R)$?
\end{exercise}

In practice, we impose additional structure to make this correspondence well-behaved:
\begin{itemize}
    \item Between $C^*$-algebras and locally compact Hausdorff spaces, via the Gelfand representation theorem—where the term “spectrum” originates.
    \item Between commutative rings and locally ringed spaces, which is foundational in algebraic geometry.
\end{itemize}
\end{remark}

\begin{example}[Hom Functor]
For a fixed object $A$ in a category $\mathcal{C}$, the \textbf{Hom functor} is defined as:
\[
h^A := \mathrm{Hom}_{\mathcal{C}}(A, -): \mathcal{C} \to \mathbf{Set},
\]
which assigns to each object $B$ the set $\mathrm{Hom}_{\mathcal{C}}(A, B)$, and to each morphism $f: B \to C$ the function:
\[
h^A(f): \mathrm{Hom}_{\mathcal{C}}(A, B) \to \mathrm{Hom}_{\mathcal{C}}(A, C), \quad g \mapsto f \circ g.
\]
Similarly, we define the contravariant version:
\[
h_A := \mathrm{Hom}_{\mathcal{C}}(-, A): \mathcal{C}^{\mathrm{op}} \to \mathbf{Set}.
\]
\end{example}

\begin{exercise}
Show that:
\begin{itemize}
    \item $f$ is a monomorphism if and only if $h^A(f)$ is injective for all $A$.
    \item $f$ is an epimorphism if and only if $h_A(f)$ is injective for all $A$.
    \item $f$ is an isomorphism if and only if both $h^A(f)$ and $h_A(f)$ are bijective for all $A$.
\end{itemize}
\end{exercise}

\begin{remark}
We can think of objects in $\mathcal{C}$ as “test objects,” and a functor as a way of encoding data about how these tests behave. For example, in a sigma model with target space $M$, let $\mathcal{C}$ be the category of space-times. For each space-time $\Sigma$, the collection of fields $\mathrm{Map}(\Sigma, M)$ defines such a functor. The fundamental question is: given such a functor, can we reconstruct the underlying “space”?
\end{remark}

\subsection{Presheaf}
  
Now we introduce the central conception for "generalized space" of the course

\begin{definition}[Presheaf]
  Let $\mathcal{C}$ be a category. A \textbf{presheaf} on $\mathcal{C}$ is a functor:
\[
F: \mathcal{C}^{\mathrm{op}} \to \mathbf{Set}
\]

That is, $F$ assigns:
\begin{itemize}
    \item To each object $U$ in $\mathcal{C}$, a set $F(U)$.
    \item To each morphism $f: V \to U$ in $\mathcal{C}$, a function:
    \[
    F(f): F(U) \to F(V)
    \]
\end{itemize}

such that:
\begin{itemize}
    \item $F(\mathrm{id}_U) = \mathrm{id}_{F(U)}$
    \item $F(g \circ f) = F(f) \circ F(g)$ for composable morphisms $f$ and $g$ in $\mathcal{C}$
\end{itemize}
\end{definition}
\begin{remark}
 Let $\PSh(\cC)=\Fun(\cC,\Set)$ be the collection of presheaf.

To understand why is notion of generalized space, let us compare to how do we define distribution as generalized function: 
\begin{itemize}
  \item We begin with some test function, Let $\mathcal{D}(\Omega)$ denotes the space of test functions, i.e., smooth functions with compact support in $\Omega$, notice that we have a pairing 
    $$ <-,->:\mathcal{D}(\Omega)\times \mathcal{D}(\Omega)\to \bR, (f,g) \mapsto \int_\Omega fg $$
    By which we can send a test function $f\in \mathcal{D}(\Omega)$ be a functional $ T_f :\mathcal{D}'(\Omega):=\mathcal{D}(\Omega)\to \bR, \phi \mapsto \int_\Omega f\phi $. Then the we define the \emph{distribution} as $ \mathcal{D}'(\Omega)$, which are generalized functions.
  \item Now we begin with category of test spaces $\cC$, we have the hom functor 
    \[
      \Hom_\cC(-,-):\cC^\op \times \cC \to \Set
    \]
    For each object $A\in \cC$ we can define presheaf $h_A: \cC^\op  \to \Set$, then we think presheaf as generalized spaces. Just like case of distribution, we have more presheaf than $h_A$.
  \item When we solve differntial equation, we first get a distribution solution, then we discuss the \emph{regularity} of distribution, i.e. locally how closed is it to test functions, to conclude the solution is an actual function.
  \item When we solve some geometric problem, for example moduli problem, we need to construct a space $M$, we can first define a (pre)sheaf, then we can discuss the \emph{repsentablity} of this (pre)sheaf, i.e. locally the sheaf is look like the test spaces, we can conclude we construct an actual space.
\end{itemize}

\end{remark}
As we have seen, for any $A\in \cC$, $h_A$ is a presheaf, such presheaf we called \emph{representable}. Just like there are non-smooth distribution, we have more presheaf than representable ones. Aside from $h_A$, now we give more concrete examples of presheaves.
\begin{example}[(Pre)sheaf of function]
  Let $X$ be a topological space, for each open set $U\in \Op(X)$, the set of continuous function $C(U)$ define a presheaf $C \in \PSh(\Op(X))$: for $U\subseteq V$, we have the restriction $C(V)\to C(U)$. In general, for any other topological space $Y$ we can have a presheaf  $C(-,Y) \in \PSh(\Op(X))$. Similarly, you can define presheaf of all kinds of function: smooth, analytic, locally constant, etc.
\end{example}
\begin{example}[(Pre)sheaf of section]
  Let $E \to X$ be a vector bundle over a topological space $X$. Define a presheaf $\mathcal{F}$ by assigning to each open set $U \subseteq X$ the set of continuous (or smooth) sections of $E$ over $U$:
    \[
    \mathcal{F}(U) = \{ s : U \to E \mid s \text{ is a section of } E \text{ over } U \}.
    \]
\end{example}
The following is one of the main example 
\begin{example}[Smooth Set]
  Let category \textbf{Cartesian space} $\Cart$ have the objects $\bR^n, n\in \mathbb{N}$, morphism $\Hom_\Cart(\bR^n,\bR^m)$ are smooth functions. We call (pre)sheaf of Cartesian space as smooth set.
\begin{itemize}
  \item \textbf{Manifold.} For any smooth manifold $M$ define a presheaf $M(\bR^n)= \Map_{\mathrm{sm}}(\bR^n, M)$ defined by smooth map. So all manifold are smooth set.
  \item \textbf{Differential forms.} Consider differential from $ \Omega^k(\bR^n)$ and its pullback along smooth functions, we have a smooth set $\Omega^k$. Show that $\Omega^k$ is not a manifold.
\end{itemize}

\end{example}

\begin{example}[Algebraic Set]
As we mention before, in algebraic geometry, we should think $\Ring^\op$ as "test spaces", Then presheaf $\mathcal{F}\in \PSh(\Ring^\op) = \Fun(\Ring,\Set)$ is nothing but a functor from $\Ring$ to $\Set$. Let us call such functor algebraic set.
  \begin{itemize}

    \item \textbf{Affine Variety}
  As we pointed out, we should study the geometry of zero set of polynomials. Let $P_i \in \bZ[x_1,\ldots,x_n] $ for $i=1,\ldots,m$ be some polynomails, now we define functor $V_P: \Ring \to \Set$

    \[
      V_{P}(R) = \{ (r_1,\ldots,r_n) \in R^n \mid \forall i, P_i(r_1,\ldots,r_n) =0 \}.
    \]
  \begin{exercise}
    Let $R_P= \bZ[x_1,\ldots,x_n]/(P_1,\ldots,P_m)$, show that $V_P= h^{R_P}$, i.e. $ V_P(R)= \Hom_\Ring(R_P, R)$. 
  \end{exercise}
\item \textbf{Projective Space}
  Let us consider a non-representable alebraic set: let functor $\mathbb{P}^n : \Ring \to \Set$ to be
    \[
      \mathbb{P}^n(R) = \{ (r_0,\ldots,r_n) \in R^{n+1} \mid \exists  (u_0,\ldots,u_n) \in R^{n+1} \text{ s.t. } \sum_{i=0}^n u_ir_i =1 \}/ \sim 
    \]
    where $\sim$ is $[\forall a\in R^{\times}, (r_0,\ldots,r_n)\sim (ar_0,\ldots,ar_n)]$.
  \begin{exercise}
    Show that this defines a functor. And compare this to projective space in differential geometry when take $R=\bR$.
  \end{exercise}
  \end{itemize}
\end{example}

As we indicated, $\PSh(\cC)$ should be a category, let us found out what morphism they should have: Since $h_A,h_B$ is thought as generalized space, it should be inherent the morphism in $\cC$. We can check that for $f:A\to B$, we have the map of set $ f\circ -: h_A(C)\to h_B(C)$ for each $C$. Nevertheless, those map should be compactible for the chose of $C$. This lead to our definition:
\begin{definition}
  
A \emph{morphism of presheaves} $\varphi : \mathcal{F} \to \mathcal{G}$ is a natural transformation between the functors $\mathcal{F}$ and $\mathcal{G}$; that is, for each object $U$ in $\mathcal{C}$, there is a function $\varphi_U : \mathcal{F}(U) \to \mathcal{G}(U)$ such that for every morphism $f : V \to U$ in $\mathcal{C}$, the following diagram commutes:
% https://q.uiver.app/#q=WzAsNCxbMCwwLCJcXGJ1bGxldCJdLFsxLDAsIlxcYnVsbGV0Il0sWzAsMSwiXFxidWxsZXQiXSxbMSwxLCJcXGJ1bGxldCJdLFswLDEsIlxcdmFycGhpX1UiXSxbMCwyLCJcXG1hdGhjYWx7Rn0oZikiLDJdLFsyLDMsIlxcdmFycGhpX1YiXSxbMSwzLCJcXG1hdGhjYWx7R30oZikiXV0=
\[\begin{tikzcd}[ampersand replacement=\&,cramped]
  \mathcal{F}(U) \&  \mathcal{G}(U) \\
\mathcal{F}(V)   \& \mathcal{G}(V)
	\arrow["{\varphi_U}", from=1-1, to=1-2]
	\arrow["{\mathcal{F}(f)}"', from=1-1, to=2-1]
	\arrow["{\mathcal{G}(f)}", from=1-2, to=2-2]
	\arrow["{\varphi_V}", from=2-1, to=2-2]
\end{tikzcd}\]

\end{definition}
\begin{exercise}
  Show that a morphism $f:A\to B$ induce a morphism of presheaf $Y(f):h_A\to h_B$. This means we have in fact a functor $ Y: \cC \to \PSh(\cC), A\mapsto h_A $. This is called \emph{Yoneda Embedding}.
\end{exercise}
\begin{exercise}*
  Let $M$ be a smooth manifold and seen as smooth set, Show that every differential form $\Omega^k(M)$ on $M$ induce a morphism of smooth sets $\Hom_{sm}(M,\Omega^k)$. Then show that is a bijection $ \Omega^k(M)\to \Hom_{sm}(M,\Omega^k)$.
\end{exercise}
\section{Yoneda Lemma}

\section{Universal Property and (Co)limit}
